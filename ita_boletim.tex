\documentclass[12pt]{iptex}
% IDIOMA
% Se necessário, inserir línguas e indicar língua principal (main)
\usepackage[main=brazil,english]{babel}
\usepackage[italic]{mathastext}
\sisetup{
    group-separator={.},
    group-minimum-digits=3,
    output-decimal-marker={,}}

% Alterando nome do TOC (adicionar em outras línguas, se necessário)
\addto\captionsbrazil{%
	\renewcommand{\contentsname}{SUMÁRIO}
	\renewcommand{\refname}{REFERÊNCIAS}
}

% BIBLIOGRAFIA
%\usepackage[abnt-emphasize=bf,alf]{abntex2cite}
% Para bibliografia ABNT com números
\usepackage[abnt-emphasize=bf,num]{abntex2cite}
\citebrackets[]

% Para outros estilos o autor deve definir o \bibliographystyle dentro do documento

% Início do documento
\begin{document}

% CAPA 
\input{./ita/tex/capa_ita_boletim.tex}

% Corpo
\input{./ita/tex/corpo_ita_boletim.tex}

\section{COMPLETUDE DOS DADOS}

\begin{figure}[htb!]
    \centering
	\captionsetup{justification=raggedright, singlelinecheck=false, width=1\textwidth}
    \caption{Gráfico de completude dos dados para o mês de agosto/2023 para a estação SP7.}
    \begin{mdframed}[
        linecolor=black,
        linewidth=1pt,
        roundcorner=10pt,
    ]
    \includegraphics[width=1.0\textwidth]{./ita/figuras/completude_SP7.png} % Substitua pelo nome da imagem e ajuste o tamanho
    %\includegraphics[width=1.0\textwidth]{./ita/figuras/completude_SP7.png}
    \end{mdframed}
    \caption*{Fonte: IPT}
    \label{fig:completude}
\end{figure}



\input{./ita/tex/tabela_ita_boletim.tex}

\input{./ita/tex/mapa_ita_boletim.tex}

% Bibliografia
\clearpage
\section{REFERÊNCIAS BIBLIOGRÁFICAS}

C. F. RICHTER, \textit{Elementary Seismology}, W. H. Freeman and Co., San Francisco, 1958, 768 pp.
%\renewcommand{\refname}{REFERÊNCIAS}
%\addcontentsline{toc}{section}{REFERÊNCIAS}
%\bibliography{ref}

\end{document}
