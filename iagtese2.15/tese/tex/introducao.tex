\chapter{Introdu��o}
\label{intro}

\section{como incluir cita��es utilizando bibTeX}

\begin{verbatim}O comando \citet{doutorado} produz:\end{verbatim}
\citet{doutorado}

\begin{verbatim}O comando \citet{mestrado} produz:\end{verbatim}
\citet{mestrado}

\begin{verbatim}O comando \citet*{artigo1} produz:\end{verbatim}
\citet*{artigo1}

\begin{verbatim}O comando \citet{artigo2} produz:\end{verbatim}
\citet{artigo2}

\begin{verbatim}O comando \citet{bethe} produz:\end{verbatim}
\citet{bethe}

\begin{verbatim}O comando \citet{livro} produz:\end{verbatim}
\citet{livro}

\begin{verbatim}O comando \citet{feynman} produz:\end{verbatim}
\citet{feynman}

\begin{verbatim}O comando \cite{salpeter} produz:\end{verbatim}
\cite{salpeter}

\section{pr�xima se��o}

\subsection{subse��o}

\subsubsection{subsubse��o}

\paragraph{par�grafo}
