\section{ATIVIDADES REALIZADAS E RESULTADOS OBTIDOS}
\label{sec:ativ_real}
\par{Neste período de estudos, foi desenvolvido e implementado um Classificador Sismológico avançado, empregando redes neurais convolucionais para a classificação de espectrogramas de eventos sismológicos. Este algoritmo foi concebido para distinguir entre eventos naturais e antropogênicos, proporcionando uma ferramenta robusta para análises sismológicas detalhadas.}

\par{O Classificador Sismológico foi desenvolvido em Python e é mantido em um repositório no GitLab, sob a colaboração entre o Laboratório de Planetologia e Geociências da Universidade de Nantes, França, e o setor de Sismologia do IPT. O código desenvolvido permite desde a aquisição até a análise das métricas de desempenho do modelo.}

%%%%%%%%%%%%%%%%%%%%%%%%%%%%%%%%%%%%%%%%%%%%%%%%%%%%%%%%%%%%%%%%%%%%%%%%%%%%%%%
\subsection{Desenvolvimento e configuração do sistema}
\label{subsec:desenvolvimento}
\par{O sistema foi estruturado para operar de maneira dinâmica e eficiente, permitindo a aplicação do algoritmo de classificação francês de forma integrada com os procedimentos de aquisição de dados sismológicos. A instalação e configuração do ambiente para o classificador foram automatizadas por meio de scripts, facilitando a reprodução e a execução em diferentes infraestruturas computacionais.}

\par{Para a coleta de dados, foi estabelecida uma pipeline que integra o download, a filtragem e o armazenamento dos dados sismológicos, utilizando catálogos do MOHO (IAG-USP) para garantir a obtenção de eventos naturais. A classificação dos eventos foi realizada considerando distintos períodos do dia para discernir entre eventos naturais e antrópicos, com uma análise adicional da forma de onda no software Snuffler.}

%%%%%%%%%%%%%%%%%%%%%%%%%%%%%%%%%%%%%%%%%%%%%%%%%%%%%%%%%%%%%%%%%%%%%%%%%%%%%%%
\subsection{Testes e validação do Classificador}
\label{subsec:testes_validacao}
\par{O classificador foi testado utilizando um conjunto de dados rigorosamente selecionado, composto por eventos rotulados como naturais por especialistas. Esta fase de testes foi crucial para validar a precisão do classificador na discriminação entre eventos antrópicos e naturais, ajustando parâmetros e refinando o modelo conforme necessário.}

\par{Os resultados dos testes foram encorajadores, mostrando uma boa capacidade do modelo em identificar corretamente a natureza dos eventos sismológicos. As métricas de desempenho, como precisão e recall, foram calculadas e apresentaram resultados satisfatórios, reforçando a eficácia do classificador desenvolvido.}

\par{Adicionalmente, foram realizados ajustes baseados nos resultados dos testes, incluindo a otimização da captura de dados e do pré-processamento, para melhorar a acurácia das classificações.}

%%%%%%%%%%%%%%%%%%%%%%%%%%%%%%%%%%%%%%%%%%%%%%%%%%%%%%%%%%%%%%%%%%%%%%%%%%%%%%%
\subsection{Implicações e futuras diretrizes}
\label{subsec:implicacoes}
\par{Os avanços alcançados com o desenvolvimento deste classificador sismológico abrem novas perspectivas para a análise de dados sismológicos no Brasil. Com a capacidade de discriminar de forma eficiente entre eventos naturais e antropogênicos, o classificador é uma ferramenta valiosa para o monitoramento ambiental e para a pesquisa geológica.}

\par{Recomenda-se a continuidade do desenvolvimento do sistema, com atualizações regulares do modelo e a integração de novas técnicas de análise de dados sismológicos. Além disso, é crucial manter a colaboração entre instituições de pesquisa nacionais e internacionais para aprimorar constantemente as capacidades de monitoramento sismológico do país.}

\par{A implementação do classificador em outras regiões e para diferentes tipos de dados sismológicos também será explorada, visando expandir sua aplicabilidade e contribuir para um conhecimento mais profundo da dinâmica sismológica regional e global.}

%%%%%%%%%%%%%%%%%%%%%%%%%%%%%%%%%%%%%%%%%%%%%%%%%%%%%%%%%%%%%%%%%%%%%%%%%%%%%%%
\subsection{Análise de Eventos em Horários Não Comerciais}
\label{subsec:nao_comerciais}


\begin{figure}[ht!]
	\captionsetup{justification=justified, singlelinecheck=false, width=1\textwidth}
    \caption{Mapa do Brasil mostrando pontos de interesse e os epicentros dos eventos classificados como detonações e sismos. Foram detectados um total de sessenta e sete (67) eventos associados a detonações no período, classificados a partir do horário de ocorrência e da forma de onda, além do plano de fogo fornecido, com magnitudes mínima e máxima de 0.4 e 3.0 MLv, respectivamente.}
    \begin{mdframed}[
        linecolor=black,
        linewidth=1pt,
        roundcorner=10pt,
    ]
    \begin{center}
    \includegraphics[width=0.8\textwidth]{/home/ggrl/projetos/ClassificadorSismologico/arquivos/figuras/mapas/mapa.png}
    \end{center}
    \end{mdframed}
    \caption*{Fonte: IPT}
\end{figure}


\par{Uma parte significativa do estudo envolveu a análise de eventos sismológicos registrados em horários não comerciais, definidos como o período entre as 23:00 UTC e 11:00 UTC. Este intervalo foi escolhido considerando as diferenças de fuso horário entre as várias regiões do Brasil, que abrangem de -3 UTC a -5 UTC. A análise focou em identificar características distintivas dos eventos naturais e antropogênicos ocorridos neste período.}

\par{Os dados foram processados e visualizados usando uma série de scripts Python desenvolvidos para filtrar, analisar e plotar informações sismológicas detalhadas. Os gráficos resultantes, como distribuições de probabilidade natural, boxplots de distância por natureza do evento e matrizes de correlação, ajudaram a ilustrar diferenças significativas nas características dos eventos registrados durante o horário não comercial.}

\par{Especificamente, a distribuição de \textit{prob\_nat} (probabilidade de um evento ser natural) para eventos naturais mostrou-se distinta daquela para eventos antropogênicos, com eventos naturais tendendo a ter valores mais altos de probabilidade. Além disso, a análise de recall por número de estações envolvidas no registro dos eventos revelou que um maior número de estações frequentemente correlaciona-se a uma classificação mais precisa entre eventos naturais e antropogênicos.}

\par{Os histogramas de recall ajustados para a hora do dia destacaram a precisão da classificação durante os horários não comerciais, refletindo a eficácia do algoritmo em identificar corretamente a natureza dos eventos sob condições variáveis de ruído ambiental e atividade humana.}

\par{Todas essas análises são cruciais para entender a dinâmica sismológica em horários menos típicos para atividades humanas, oferecendo insights sobre a influência de fatores naturais isolados das interferências antrópicas. Os resultados estão detalhados nos Apêndices A e B, onde figuras e tabelas fornecem uma representação visual e quantitativa das descobertas.}

\par{A validação destes resultados foi facilitada pela utilização de scripts automatizados que permitiram uma reprodutibilidade eficiente e a geração de outputs consistentes para análises subsequentes e revisões de estudo.}

\par{Este foco em eventos durante períodos não comerciais não apenas enriquece a compreensão da sismicidade natural mas também aprimora as metodologias de monitoramento e análise sismológica em condições controladas de ruído.}


                    \begin{figure}[H]
                        \centering
                        \includegraphics[width=1.0\textwidth]{/home/ipt/projetos/ClassificadorSismologico/arquivos/figuras/pos_processo/hist_ev_hour_recall.png}
                        \caption{Hist ev hour recall}
                        \label{fig:hist_ev_hour_recall}
                    \end{figure}
                

                    \begin{figure}[H]
                        \centering
                        \includegraphics[width=1.0\textwidth]{/home/ipt/projetos/ClassificadorSismologico/arquivos/figuras/pos_processo/dist_mean_snrs_recall.png}
                        \caption{Dist mean snrs recall}
                        \label{fig:dist_mean_snrs_recall}
                    \end{figure}
                

                    \begin{figure}[H]
                        \centering
                        \includegraphics[width=1.0\textwidth]{/home/ipt/projetos/ClassificadorSismologico/arquivos/figuras/pos_processo/dist_snrs.png}
                        \caption{Dist snrs}
                        \label{fig:dist_snrs}
                    \end{figure}
                

                    \begin{figure}[H]
                        \centering
                        \includegraphics[width=1.0\textwidth]{/home/ipt/projetos/ClassificadorSismologico/arquivos/figuras/pos_processo/boxplot_IM.png}
                        \caption{Boxplot im}
                        \label{fig:boxplot_IM}
                    \end{figure}
                

                    \begin{figure}[H]
                        \centering
                        \includegraphics[width=1.0\textwidth]{/home/ipt/projetos/ClassificadorSismologico/arquivos/figuras/pos_processo/dist_ev_distance_rel_freq.png}
                        \caption{Dist ev distance rel freq}
                        \label{fig:dist_ev_distance_rel_freq}
                    \end{figure}
                

                    \begin{figure}[H]
                        \centering
                        \includegraphics[width=1.0\textwidth]{/home/ipt/projetos/ClassificadorSismologico/arquivos/figuras/pos_processo/boxplot_BX.png}
                        \caption{Boxplot bx}
                        \label{fig:boxplot_BX}
                    \end{figure}
                

                    \begin{figure}[H]
                        \centering
                        \includegraphics[width=1.0\textwidth]{/home/ipt/projetos/ClassificadorSismologico/arquivos/figuras/pos_processo/boxplot_BL.png}
                        \caption{Boxplot bl}
                        \label{fig:boxplot_BL}
                    \end{figure}
                

                    \begin{figure}[H]
                        \centering
                        \includegraphics[width=1.0\textwidth]{/home/ipt/projetos/ClassificadorSismologico/arquivos/figuras/pos_processo/hist_mean_snrs_recall.png}
                        \caption{Hist mean snrs recall}
                        \label{fig:hist_mean_snrs_recall}
                    \end{figure}
                

                    \begin{figure}[H]
                        \centering
                        \includegraphics[width=1.0\textwidth]{/home/ipt/projetos/ClassificadorSismologico/arquivos/figuras/pos_processo/dist_prob_nat_recall.png}
                        \caption{Dist prob nat recall}
                        \label{fig:dist_prob_nat_recall}
                    \end{figure}
                

                    \begin{figure}[H]
                        \centering
                        \includegraphics[width=1.0\textwidth]{/home/ipt/projetos/ClassificadorSismologico/arquivos/figuras/pos_processo/dist_ev_cat_mag.png}
                        \caption{Dist ev cat mag}
                        \label{fig:dist_ev_cat_mag}
                    \end{figure}
                

                    \begin{figure}[H]
                        \centering
                        \includegraphics[width=1.0\textwidth]{/home/ipt/projetos/ClassificadorSismologico/arquivos/figuras/pos_processo/boxplot_rede.png}
                        \caption{Boxplot rede}
                        \label{fig:boxplot_rede}
                    \end{figure}
                

                    \begin{figure}[H]
                        \centering
                        \includegraphics[width=1.0\textwidth]{/home/ipt/projetos/ClassificadorSismologico/arquivos/figuras/pos_processo/dist_snrs_recall.png}
                        \caption{Dist snrs recall}
                        \label{fig:dist_snrs_recall}
                    \end{figure}
                

                    \begin{figure}[H]
                        \centering
                        \includegraphics[width=1.0\textwidth]{/home/ipt/projetos/ClassificadorSismologico/arquivos/figuras/pos_processo/boxplot_XV.png}
                        \caption{Boxplot xv}
                        \label{fig:boxplot_XV}
                    \end{figure}
                

                    \begin{figure}[H]
                        \centering
                        \includegraphics[width=1.0\textwidth]{/home/ipt/projetos/ClassificadorSismologico/arquivos/figuras/pos_processo/dist_mean_snrs_prob_nat.png}
                        \caption{Dist mean snrs prob nat}
                        \label{fig:dist_mean_snrs_prob_nat}
                    \end{figure}
                

                    \begin{figure}[H]
                        \centering
                        \includegraphics[width=1.0\textwidth]{/home/ipt/projetos/ClassificadorSismologico/arquivos/figuras/pos_processo/hist_ev_distance.png}
                        \caption{Hist ev distance}
                        \label{fig:hist_ev_distance}
                    \end{figure}
                

                    \begin{figure}[H]
                        \centering
                        \includegraphics[width=1.0\textwidth]{/home/ipt/projetos/ClassificadorSismologico/arquivos/figuras/pos_processo/boxplot_VL.png}
                        \caption{Boxplot vl}
                        \label{fig:boxplot_VL}
                    \end{figure}
                

                    \begin{figure}[H]
                        \centering
                        \includegraphics[width=1.0\textwidth]{/home/ipt/projetos/ClassificadorSismologico/arquivos/figuras/pos_processo/dist_mean_snrs.png}
                        \caption{Dist mean snrs}
                        \label{fig:dist_mean_snrs}
                    \end{figure}
                

                    \begin{figure}[H]
                        \centering
                        \includegraphics[width=1.0\textwidth]{/home/ipt/projetos/ClassificadorSismologico/arquivos/figuras/pos_processo/hist_mean_snrs.png}
                        \caption{Hist mean snrs}
                        \label{fig:hist_mean_snrs}
                    \end{figure}
                

                    \begin{figure}[H]
                        \centering
                        \includegraphics[width=1.0\textwidth]{/home/ipt/projetos/ClassificadorSismologico/arquivos/figuras/pos_processo/dist_ev_num_stations_absoluto.png}
                        \caption{Dist ev num stations absoluto}
                        \label{fig:dist_ev_num_stations_absoluto}
                    \end{figure}
                

                    \begin{figure}[H]
                        \centering
                        \includegraphics[width=1.0\textwidth]{/home/ipt/projetos/ClassificadorSismologico/arquivos/figuras/pos_processo/mean_snrs_3_prob_nat.png}
                        \caption{Mean snrs 3 prob nat}
                        \label{fig:mean_snrs_3_prob_nat}
                    \end{figure}
                

                    \begin{figure}[H]
                        \centering
                        \includegraphics[width=1.0\textwidth]{/home/ipt/projetos/ClassificadorSismologico/arquivos/figuras/pos_processo/dist_ev_cat_mag_recall.png}
                        \caption{Dist ev cat mag recall}
                        \label{fig:dist_ev_cat_mag_recall}
                    \end{figure}
                

                    \begin{figure}[H]
                        \centering
                        \includegraphics[width=1.0\textwidth]{/home/ipt/projetos/ClassificadorSismologico/arquivos/figuras/pos_processo/dist_ev_num_stations_recall.png}
                        \caption{Dist ev num stations recall}
                        \label{fig:dist_ev_num_stations_recall}
                    \end{figure}
                

                    \begin{figure}[H]
                        \centering
                        \includegraphics[width=1.0\textwidth]{/home/ipt/projetos/ClassificadorSismologico/arquivos/figuras/pos_processo/boxplot_dist.png}
                        \caption{Boxplot dist}
                        \label{fig:boxplot_dist}
                    \end{figure}
                

                    \begin{figure}[H]
                        \centering
                        \includegraphics[width=1.0\textwidth]{/home/ipt/projetos/ClassificadorSismologico/arquivos/figuras/pos_processo/hist_snrs_recall.png}
                        \caption{Hist snrs recall}
                        \label{fig:hist_snrs_recall}
                    \end{figure}
                

                    \begin{figure}[H]
                        \centering
                        \includegraphics[width=1.0\textwidth]{/home/ipt/projetos/ClassificadorSismologico/arquivos/figuras/pos_processo/region_corr.png}
                        \caption{Region corr}
                        \label{fig:region_corr}
                    \end{figure}
                

                    \begin{figure}[H]
                        \centering
                        \includegraphics[width=1.0\textwidth]{/home/ipt/projetos/ClassificadorSismologico/arquivos/figuras/pos_processo/boxplot_XC.png}
                        \caption{Boxplot xc}
                        \label{fig:boxplot_XC}
                    \end{figure}
                

                    \begin{figure}[H]
                        \centering
                        \includegraphics[width=1.0\textwidth]{/home/ipt/projetos/ClassificadorSismologico/arquivos/figuras/pos_processo/boxplot_ON.png}
                        \caption{Boxplot on}
                        \label{fig:boxplot_ON}
                    \end{figure}
                

                    \begin{figure}[H]
                        \centering
                        \includegraphics[width=1.0\textwidth]{/home/ipt/projetos/ClassificadorSismologico/arquivos/figuras/pos_processo/boxplot_Y4.png}
                        \caption{Boxplot y4}
                        \label{fig:boxplot_Y4}
                    \end{figure}
                

                    \begin{figure}[H]
                        \centering
                        \includegraphics[width=1.0\textwidth]{/home/ipt/projetos/ClassificadorSismologico/arquivos/figuras/pos_processo/mean_snrs_1_prob_nat.png}
                        \caption{Mean snrs 1 prob nat}
                        \label{fig:mean_snrs_1_prob_nat}
                    \end{figure}
                

                    \begin{figure}[H]
                        \centering
                        \includegraphics[width=1.0\textwidth]{/home/ipt/projetos/ClassificadorSismologico/arquivos/figuras/pos_processo/boxplot_BR.png}
                        \caption{Boxplot br}
                        \label{fig:boxplot_BR}
                    \end{figure}
                

                    \begin{figure}[H]
                        \centering
                        \includegraphics[width=1.0\textwidth]{/home/ipt/projetos/ClassificadorSismologico/arquivos/figuras/pos_processo/boxplot_NB.png}
                        \caption{Boxplot nb}
                        \label{fig:boxplot_NB}
                    \end{figure}
                

                    \begin{figure}[H]
                        \centering
                        \includegraphics[width=1.0\textwidth]{/home/ipt/projetos/ClassificadorSismologico/arquivos/figuras/pos_processo/hist_hora.png}
                        \caption{Hist hora}
                        \label{fig:hist_hora}
                    \end{figure}
                

                    \begin{figure}[H]
                        \centering
                        \includegraphics[width=1.0\textwidth]{/home/ipt/projetos/ClassificadorSismologico/arquivos/figuras/pos_processo/boxplot_GT.png}
                        \caption{Boxplot gt}
                        \label{fig:boxplot_GT}
                    \end{figure}
                

%%%%%%%%%%%%%%%%%%%%%%%%%%%%%%%%%%%%%%%%%%%%%%%%%%%%%%%%%%%%%%%%%%%%%%%%%%%%%%%
\subsection{Documentação e divulgação}
\label{subsec:documentacao}
\par{Todos os códigos e algoritmos desenvolvidos estão devidamente documentados e disponíveis publicamente no repositório do projeto. A documentação inclui guias de instalação, configuração e utilização do sistema, permitindo que outros pesquisadores e técnicos possam utilizar e adaptar o classificador para suas necessidades específicas.}

\par{Os resultados obtidos e as metodologias empregadas foram submetidos para publicação em periódicos especializados e apresentados em conferências nacionais e internacionais, contribuindo para a disseminação do conhecimento e das inovações desenvolvidas no âmbito deste projeto.}

\par{Continuará sendo dada ênfase à formação de parcerias estratégicas e ao engajamento da comunidade científica, visando fortalecer a rede de pesquisa em sismologia no Brasil e promover o uso de tecnologias avançadas na análise de fenômenos sismológicos.}

