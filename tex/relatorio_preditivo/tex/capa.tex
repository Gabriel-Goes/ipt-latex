% Parâmetros
<<<<<<< HEAD
\tipo{RELATÓRIO DE MÉTRICAS}
\cancelaDoc{Relatório n.º} % Descomente para opção cancela e substitui
\docNum{II}
\data{14 de dezembro de 2023}
\titulo{Análise de métricas dos resultados obtidos para classificação dos eventos rede sismológica brasileira - RSBR.}
=======
\docNum{I}
\tipo{RELATÓRIO DE MÉTRICAS DE PREDIÇÃO}
\cancelaDoc{Relatório n.º 1} % Descomente para opção cancela e substitui
\data{28 de maio de 2024}
\titulo{Análise de métricas obtidas para classificação de eventos da Rede Sismológica Brasileira.}
>>>>>>> 04f97f3c907ac7cd5ebd511636a49c75a2af8270
\unidade{Cidades Infraestruturas e Meio Ambiente}{CIMA}
\lab{Seção de Obras Civis - SOC}
\periodo{2012}{2024}

% Inserir capa
\capa

% Inserir Resumo e Tabela de Conteúdo
\pagestyle{tipo_pre}
\resumo{
        Neste Relatório são apresentados os resultados das métricas obtidas
    para a classificação dos eventos registrados pela Rede Sismológica
    Brasileira. O objetivo é avaliar a eficiência do algoritmo de classificação
    automática de eventos sísmicos por redes neurais convolucionais
    desenvolvido na pelo Laboratório de Planetologia e Geociências da
    Universidade de Nantes, França. Este algoritmo foi  treinado com dados
    sísmicos da rede sismológico francesa foram testados com dados tanto desta
    mesma rede como da rede estado-unidense obtendo resultados satisfatórios.
    Neste relatório, são apresentados os resultados obtidos para a rede
    brasileira e as métricas de avaliação do desempenho do algoritmo.
    Para isso, foi desenvolvido um sistema que automatiza a aquisição e 
    preprocessamento dos dados sísmicos que serão inseridos na primeira
    camada da rede neural.
} {sismologia, aprendizagem de máquina, redes neurais convolucionais.}
\tableofcontents

\pagebreak
\renewcommand{\thepage}{\arabic{page}}
