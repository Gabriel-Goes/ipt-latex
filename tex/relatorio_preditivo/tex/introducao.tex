\section{Introdução}

\subsection{Contextualização}
\par{
    Este Relatório integra o estudo sismológico em desenvolvimento pela equipe de sismologia do SOC-IPT com os desenvolvidos pelo Laboratório de Planetologia e Geociências da Universidade de Nantes que produziu um algoritmo de classificação supervisionada de eventos sismológicos com redes neurais artificiais convolucionais. Este algoritmo foi treinado com dados de sismos na região metropolitana francesa obtendo acurácias maiores que 95\% na classificação entre eventos naturais e antropogênicos, tanto em território francês quanto em um teste feito com dados do estado norteamericano, Utah.

    A rede neural desenvolvida se apresenta com an
}

\subsection{Objetivo}
\par{
    O objetivo deste trabalho é apresentar os resultados obtidos com a classificação dos sismos do catálogo de eventos disponibilizado pela MOHO-IAG/IEE e aferir a eficácia deste algoritmo em classificar eventos em uma rede esparsa como a brasileira e em um contexto geológico distindo do conjunto de treino do modelo.
}

\subsection{Pré-processamento dos dados}

\par{
    Para iniciar o projeto foi adquirido o catalgo de eventos sismológicos através do serviço de dados da MOHO-IAG/IEE utilizando a página web disponível em \url{http://www.moho.iag.usp.br/rq/event} com o filtro por região de um polígono quadrilátero que inscreve a área do território continental brasileiro adicionado de um buffer de 400 km. Com este processo foram obtidos 10.173 eventos sismológicos de fevereiro de 1903 a maio de 2024. que, após o tratamento dos dados, foram reduzidos para 2594 eventos, dos quais, todos foram rotulados como naturais.
}

\par{
    O tratamento consistiu em adicionar um limite de idade em janeiro de 2010 e requerindo apenas os eventos que intersectam a gemeotria criada a partir do buffer de 400km do território continental brasileiro como aprenseta a Figura \ref{fig:1:map}. Além desta filtragem, nesta etapa de pré-processamento do catalogo, foram verificados os valores de profundidade dos eventos filtrados, e foi observado que praticamnente todos ocorreram com menos de 10km de profundidade, com excessão dos sismos na fronteira com Peru e Bolívia que atingem 600 km.
}

% manual_mapa.tex               
\begin{figure}[htb]
    \centering
    \subfloat[Catálogo bruto.\label{fig:1:map1}]{
        \resizebox{0.45\textwidth}{!}{\includegraphics{/home/ipt/projetos/ClassificadorSismologico/arquivos/figuras/mapas/mapa_eventos_bruto.png}}
    }
    \hfill
    \subfloat[Catálogo tratado.\label{fig:1:map2}]{
        \resizebox{0.45\textwidth}{!}{\includegraphics{/home/ipt/projetos/ClassificadorSismologico/arquivos/figuras/mapas/mapa_eventos_tratado.png}}
    }
    \caption{Distribuição dos eventos por profundidade presentes no catálogo bruto\it{(a)} e tratado(b).}
    \label{fig:1:map}
\end{figure}




\begin{figure}[htb]
    \centering
    \subfloat[Catálogo completo
    .\label{fig:2:prof1}]{
        \resizebox{0.45\textwidth}{!}{\includegraphics{/home/ipt/projetos/ClassificadorSismologico/arquivos/figuras/pre_processa/hist_profundidade_completo.png}}
    }
    \hfill
    \subfloat[Catálogo tratado.\label{fig:2:prof2}]{
        \resizebox{0.45\textwidth}{!}{\includegraphics{/home/ipt/projetos/ClassificadorSismologico/arquivos/figuras/pre_processa/hist_profundidade_filtrado.png}}
    }
    \caption{Distribuição dos eventos por profundidade presentes no catálogo completo\it{(a)} e filtrado(b).}
    \label{fig:2:prof}
\end{figure}




\par{
    Adicionado à isto, com esta análise da distribuição espacial dos eventos, foi possível observar que muitos sismos ocorrem a uma distância maior que 20 km da costa, que adicionado à profundidae, pode compor um conjunto de proxies que segmentam com maior certeza de que são eventos naturais. Também foi possível criar um terceiro parâmetro para selecionar eventos naturais, sendo os epicentros localizados em regiões de alta densidade florestal e regiões metropolitanas sem ocorrência de mineração de rochas duras.
}




\begin{figure}[htb]
    \centering
    \subfloat[Catálogo completo.\label{fig:1:hora1}]{
        \resizebox{0.45\textwidth}{!}{\input{/home/ipt/projetos/ClassificadorSismologico/arquivos/figuras/pre_processa/hist_hora_completo.pgf}}
    }
    \hfill
    \subfloat[Catálogo tratado.\label{fig:1:hora2}]{
        \resizebox{0.45\textwidth}{!}{\input{/home/ipt/projetos/ClassificadorSismologico/arquivos/figuras/pre_processa/hist_hora_filtrado.pgf}}
    }
    \caption{Distribuição dos eventos por hora do dia presentes no catálogo completo(a) e filtrado(b).}
    \label{fig:1:hora}
\end{figure}


