\section{CONSIDERAÇÕES FINAIS}
\label{sec:consid_finais}

\par{Para o monitoramento sismológico realizado no período de 01.12.2022 a 30.06.2023, tem-se que:}

\begin{itemize}
    \item Em síntese, o funcionamento da Estação SP7 pôde ser considerado satisfatório. Um detalhamento maior do funcionamento da estação para o período englobado por este relatório pode ser obtido analisando-se os boletins sísmicos no Anexo A, que contêm os gráficos de completeza diários para cada mês no período.
    \item Na área de influência do empreendimento foram registrados sessenta e sete (67) desmontes com magnitudes entre 0.4 e 3.0 (MLv), sendo quatorze (19) destes relacionados a detonações nas proximidades das pedreiras Azza e Daclande, e os restantes em outras áreas (Figura 2, Apêndice A).
    \item Foram detectados três pequenos sismos locais próximos à estação SP7, com magnitudes entre 0.2 e 0.9 MLv. O maior destes pequenos eventos foi registrado em 2023-05-26 18:31:02 (UTC).
    \item Foi detectado um sismo regional natural no território brasileiro, próximo à cidade de Iguape – SP, em 2023-06-16 11:22:00 (UTC). O sismo teve magnitude 4.0 mR .
    \item Durante o monitoramento sismológico local efetuado com a Estação SP7 não foram registrados sismos induzidos oriundos da operação do reservatório. 
    \item A orientação e procedimentos apresentados no Relatório IPT no 115 463-205 – “Análise dos registros obtidos entre 1º de junho e 30 de novembro de 2009, na Estação Sismológica SP7, SC”, emitido em janeiro de 2010, e no Relatório IPT no 120 081-205 – “Análise dos registros obtidos entre 1º de junho e 30 de novembro de 2010 na Estação Sismológica SP7, Salto Pilão, SC e síntese das atividades e dos resultados do monitoramento sismológico”, emitido em janeiro de 2011, quanto à ocorrência de provável tremor de terra sentido pela população local ou ocorrências anômalas na área do empreendimento, devem ser mantidos.
\end{itemize}

\par{O monitoramento instrumental possibilita determinar o epicentro, quantificar o tamanho (a magnitude), definir a origem do evento e, se for o caso, em função da análise do comportamento espaço-temporal da atividade, tomar medidas mitigatórias.} 
\par{Assim, pelos resultados do monitoramento sismológico realizado com a Estação SP7, em função das características operacionais do registrador-sismômetro e de sua localização, considera-se de muita valia manter esta estação em funcionamento para dar continuidade no conhecimento da sismicidade local e regional.}
\par{No monitoramento realizado não foi observada a ocorrência de evento sísmico associado à implementação do Empreendimento da UHE Salto Pilão. A continuidade do monitoramento sismológico na área do empreendimento permitirá acompanhar eventual ocorrência local.}
\par{Mantidas as atuais características da sismicidade, local e regional, para a continuidade do monitoramento sismológico devem ser mantidos os atuais procedimentos adotados para operação, coleta e análise dos dados, com periodicidade mensal. Além disso, continuam válidas as orientações a serem adotadas no caso de ocorrer evento local sentido pela população e de eventuais anomalias na obra tais como desplacamentos de rochas e estampidos associados.}

