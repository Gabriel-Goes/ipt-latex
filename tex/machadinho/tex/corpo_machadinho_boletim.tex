\pagestyle{geral}
% Corpo do documento
\section{ÚLTIMOS RELATÓRIOS TÉCNICOS}
\label{sec:ultimos_relatorios}
\begin{itemize}
	\item "Relatório Síntese UHMC 2023: Monitoramento sismológico na área do reservatório de Aproveitamento Hidrelétrico de Machadinho, SC/RS.", emitido em abril de 2023. 
    \item Relatório IPT Nº 205 166 666-1 -- "Análise dos registros obtidos entre 01 de dezembro de 2019 e 31 de dezembro de 2021 na rede Sismológica de Itá/Machadinho, RSIM, SC/RS.", emitido em novembro de 2022.  
\end{itemize}

\section{ATIVIDADES REALIZADAS}
\label{sec:atividade}
\begin{itemize}
    \item Encaminhamento do boletim sísmico nº 29/36-2024, setembro-2023;
    \item Coleta de dados em 03/10/2023 e envio dos mesmos para análise no IPT;
    \item Processamento preliminar dos dados do período (01/09/2023 a 30/09/2023);
    \item Para o período, não houve acesso ao plano de fogo da obra PCH Tupitinga e das pedreiras Kerbermix e PlanaTerra. A Pedreira Engenhos forneceu plano de fogo;
    \item Análise preliminar do período que inclui as coletas BCM223244 (03/08/2023 a 01/09/2023), BCM223276 (01/09/2023 a 03/10/2023) e MC923276 (26/09/2023 a 03/10/2023); e   
    \item Elaboração do gráfico de completeza dos dados, tabela contendo os registros de eventos/detonações detectados e mapa da região indicando os epicentros localizados;
\end{itemize}

\section{RESULTADOS}
\label{sec:resultados}
\begin{itemize}
	\item Foram detectados 9 sismos induzidos na região do empreendimento de Machadinho durante o período, sendo 5 próximos à estação BCM2 e 4 próximos à estação MC9. Todos eventos tiveram magnitudes entre -1.0 e 0.0 MLv, eventos pequenos. O maior evento foi detectado em 2023-09-29 10:01:22 (UTC), próximo à estação MC9, com magnitude 0.0 MLv. Não há relatos de eventos que tenham sido sentidos pela população local.  
    \item Foram detectados 2 (dois) desmontes durante o período, sendo o de maior magnitude em 2023-09-06 15:18:24 (UTC) com magnitude 1.7 MLv. Todos desmontes detectados ocorreram longe da região do reservatório (incluindo o de maior magnitude). 
    \item Os parâmetros sísmicos dos eventos detectados são detalhados na Tabela 1. O gráfico de completeza dos dados para as estações BCM2 e MC9 no mês de setembro/2023 é mostrado na Figura 1. Os epicentros dos eventos detectados são mostrados na Figura 2. 
    \item Não foram detectados sismos naturais locais, regionais e/ou telessismos em território brasileiro durante o período englobado por esse relatório. 
    \item O funcionamento das estações BCM2 e MC9 foi adequado no mês de setembro/2023. A estação MC9 foi reinstalada por equipe ténica do IPT e da ENGIE em campo no dia 26/09/2023. O antigo digitalizador da estação foi devolvido à sede do Consórcio Machadinho em Piratuba -- SC. O técnico de campo do consórcio foi treinado para a aquisição dos dados utilizando os novos equipamentos. A estação mostrou funcionamento normal durante a primeira coleta. Sendo assim, a estação MC9 se encontra reinstalada e operacional.
\end{itemize}
    

\section{CONSIDERAÇÕES}
\label{sec:consideracoes}
Continuam válidas as considerações e orientações anteriores a respeito das medidas a serem tomadas em caso ocorrência de um sismo local sentido pela população, i.e., coletar os relatos da população local através de questionários macrossísmicos, contactar a defesa civil para avaliar possíveis danos em estruturas e fornecer orientações e informações à população. 

\assinaturaLucas
\clearpage
\newpage
