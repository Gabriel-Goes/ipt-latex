\documentclass[12pt]{iptex}
% IDIOMA
% Se necessário, inserir línguas e indicar língua principal (main)
\usepackage[main=brazil,english]{babel}
\usepackage[italic]{mathastext}
\usepackage{siunitx}
\sisetup{
    group-separator={.},
    group-minimum-digits=3,
    output-decimal-marker={,}}

% Alterando nome do TOC (adicionar em outras línguas, se necessário)
\addto\captionsbrazil{%
	\renewcommand{\contentsname}{SUMÁRIO}
	\renewcommand{\refname}{REFERÊNCIAS}
}
\addto\captionsenglish{%
	\renewcommand{\contentsname}{CONTENTS}
	\renewcommand{\refname}{REFERENCES}
}
% BIBLIOGRAFIA
\usepackage[abnt-emphasize=bf,alf]{abntex2cite}

% Início do documento
\begin{document}

% CAPA 
% Parâmetros
\docNum{I}
\tipo{RELATÓRIO DE MÉTRICAS DE PREDIÇÃO}
\cancelaDoc{Relatório n.º 1} % Descomente para opção cancela e substitui
\data{28 de maio de 2024}
\titulo{Análise de métricas obtidas para classificação de eventos da Rede Sismológica Brasileira.}
\unidade{Cidades Infraestruturas e Meio Ambiente}{CIMA}
\lab{Seção de Obras Civis - SOC}
\periodo{2012}{2024}

% Inserir capa
\capa

% Inserir Resumo e Tabela de Conteúdo
\pagestyle{tipo_pre}
\resumo{Neste Relatório são apresentados os resultados do monitoramento sismológico efetuado na área da Usina Hidrelétrica Salto Pilão, por meio da Estação Sismológica SP7, no período entre 01.12.2022 e 30.06.2023, permitindo acompanhar a sismicidade local e orientar a adoção de eventuais medidas mitigadoras. Durante o monitoramento sismológico local efetuado, a Estação SP7 registrou sessenta e sete (67) desmontes em obras/pedreiras na região. No período de referência do presente relatório não foi observada a ocorrência de evento sísmico induzido pela implementação do Empreendimento da UHE Salto Pilão. Foram detectados 4 sismos naturais, sendo 3 destes sismos locais próximos à estação SP7, e um evento regional próximo à cidade de Iguape – SP, no estado de São Paulo em 2023-06-16 11:22:00 (UTC) com magnitude 4.0 mR. Ressalta-se a contribuição que este monitoramento sismológico está dando para a confirmação e a determinação dos parâmetros de eventos ocorridos no território brasileiro, em especial aqueles com epicentros nos estados de Santa Catarina e Rio Grande do Sul, e regiões vizinhas. Assim, recomenda-se que a Estação SP7 seja mantida em funcionamento, possibilitando dar continuidade ao melhor conhecimento da sismicidade local e regional, além do necessário acompanhamento da operação da Usina Hidrelétrica Salto Pilão.}{Sismologia; sismicidade; Salto Pilão; sismos induzidos; sismos naturais; detonações; UHE Salto Pilão.}
\tableofcontents

\pagebreak
\renewcommand{\thepage}{\arabic{page}}


% Corpo
\pagestyle{tipo_doc}
\setcounter{page}{1}
% Corpo do documento
% INTRODUÇÃO
\section{Introdução}

\subsection{Contextualização}
\par{
    Este Relatório integra o estudo sismológico em desenvolvimento pela equipe de sismologia do SOC-IPT com os desenvolvidos pelo Laboratório de Planetologia e Geociências da Universidade de Nantes que produziu um algoritmo de classificação supervisionada de eventos sismológicos com redes neurais artificiais convolucionais. Este algoritmo foi treinado com dados de sismos na região metropolitana francesa obtendo acurácias maiores que 95\% na classificação entre eventos naturais e antropogênicos, tanto em território francês quanto em um teste feito com dados do estado norteamericano, Utah.
}

\subsection{Objetivo}
\par{
    O objetivo deste trabalho é apresentar os resultados obtidos com a classificação dos sismos do catálogo de eventos disponibilizado pela MOHO-IAG/IEE e aferir a eficácia deste algoritmo em classificar eventos em uma rede esparsa como a brasileira e em um contexto geológico distindo do conjunto de treino do modelo.
}

\subsection{Pré-processamento dos dados}

\par{
    Para iniciar o projeto foi adquirido o catalgo de eventos sismológicos através do serviço de dados da MOHO-IAG/IEE utilizando a página web disponível em \url{http://www.moho.iag.usp.br/rq/event} com o filtro por região de um polígono quadrilátero que inscreve a área do território continental brasileiro adicionado de buffer de 400 km. O com este processo foram obtidos 10.173 eventos sismológicos de fevereiro de 1903 a maio de 2024. que, após o tratamento dos dados, foram reduzidos para 2594 eventos, dos quais, todos foram rotulados como naturais.
}

\par{
    O tratamento consistiu em adicionar um limite de idade em janeiro de 2010 e requerindo apenas os eventos que intersectam a gemeotria criada a partir do buffer de 400km do território continental brasileiro como aprenseta a Figura \ref{fig:1:map}. Além desta filtragem, nesta etapa de pré-processamento do catalogo, foram verificados os valores de profundidade dos eventos filtrados, e foi observado que praticamnente todos ocorreram com menos de 10km de profundidade, com excessão dos sismos na fronteira com Peru e Bolívia que atingem 600 km.
}

% manual_mapa.tex               
\begin{figure}[htb]
    \centering
    \subfloat[Catálogo bruto.\label{fig:1:map1}]{
        \resizebox{0.45\textwidth}{!}{\includegraphics{/home/ipt/projetos/ClassificadorSismologico/arquivos/figuras/mapas/mapa_eventos_bruto.png}}
    }
    \hfill
    \subfloat[Catálogo tratado.\label{fig:1:map2}]{
        \resizebox{0.45\textwidth}{!}{\includegraphics{/home/ipt/projetos/ClassificadorSismologico/arquivos/figuras/mapas/mapa_eventos_tratado.png}}
    }
    \caption{Distribuição dos eventos por profundidade presentes no catálogo bruto\it{(a)} e tratado(b).}
    \label{fig:1:map}
\end{figure}




\begin{figure}[htb]
    \centering
    \subfloat[Catálogo completo
    .\label{fig:2:prof1}]{
        \resizebox{0.45\textwidth}{!}{\includegraphics{/home/ipt/projetos/ClassificadorSismologico/arquivos/figuras/pre_processa/hist_profundidade_completo.png}}
    }
    \hfill
    \subfloat[Catálogo tratado.\label{fig:2:prof2}]{
        \resizebox{0.45\textwidth}{!}{\includegraphics{/home/ipt/projetos/ClassificadorSismologico/arquivos/figuras/pre_processa/hist_profundidade_filtrado.png}}
    }
    \caption{Distribuição dos eventos por profundidade presentes no catálogo completo\it{(a)} e filtrado(b).}
    \label{fig:2:prof}
\end{figure}




\par{
    Adicionado à isto, com esta análise da distribuição espacial dos eventos, foi possível observar que muitos sismos ocorrem a uma distância maior que 20 km da costa, que adicionado à profundidae, pode compor um conjunto de proxies que segmentam com maior certeza de que são eventos naturais. Também foi possível criar um terceiro parâmetro para selecionar eventos naturais, sendo os epicentros localizados em regiões de alta densidade florestal e regiões metropolitanas sem ocorrência de mineração de rochas duras.
}



               
\begin{figure}[htb]
    \centering
    \subfloat[Catálogo bruto.\label{fig:1:hora1}]{
        \resizebox{0.45\textwidth}{!}{\input{/home/ipt/projetos/ClassificadorSismologico/arquivos/figuras/pre_processa/hist_hora_bruto.pgf}}
    }
    \hfill
    \subfloat[Catálogo tratado.\label{fig:1:hora2}]{
        \resizebox{0.45\textwidth}{!}{\input{/home/ipt/projetos/ClassificadorSismologico/arquivos/figuras/pre_processa/hist_hora_tratado.pgf}}
    }
    \caption{Distribuição dos eventos por hora do dia presentes no catálogo bruto(a) e tratado(b).}
    \label{fig:1:hora}
\end{figure}




% ATIVIDADES EXECUTADAS E RESULTADOS OBTIDOS
\section{ATIVIDADES REALIZADAS E RESULTADOS OBTIDOS}
\label{sec:ativ_real}
\par{Neste período de estudos, foi desenvolvido e implementado um Classificador Sismológico avançado, empregando redes neurais convolucionais para a classificação de espectrogramas de eventos sismológicos. Este algoritmo foi concebido para distinguir entre eventos naturais e antropogênicos, proporcionando uma ferramenta robusta para análises sismológicas detalhadas.}

\par{O Classificador Sismológico foi desenvolvido em Python e é mantido em um repositório no GitLab, sob a colaboração entre o Laboratório de Planetologia e Geociências da Universidade de Nantes, França, e o setor de Sismologia do IPT. O código desenvolvido permite desde a aquisição até a análise das métricas de desempenho do modelo.}

%%%%%%%%%%%%%%%%%%%%%%%%%%%%%%%%%%%%%%%%%%%%%%%%%%%%%%%%%%%%%%%%%%%%%%%%%%%%%%%
\subsection{Desenvolvimento e configuração do sistema}
\label{subsec:desenvolvimento}
\par{O sistema foi estruturado para operar de maneira dinâmica e eficiente, permitindo a aplicação do algoritmo de classificação francês de forma integrada com os procedimentos de aquisição de dados sismológicos. A instalação e configuração do ambiente para o classificador foram automatizadas por meio de scripts, facilitando a reprodução e a execução em diferentes infraestruturas computacionais.}

\par{Para a coleta de dados, foi estabelecida uma pipeline que integra o download, a filtragem e o armazenamento dos dados sismológicos, utilizando catálogos do MOHO (IAG-USP) para garantir a obtenção de eventos naturais. A classificação dos eventos foi realizada considerando distintos períodos do dia para discernir entre eventos naturais e antrópicos, com uma análise adicional da forma de onda no software Snuffler.}

%%%%%%%%%%%%%%%%%%%%%%%%%%%%%%%%%%%%%%%%%%%%%%%%%%%%%%%%%%%%%%%%%%%%%%%%%%%%%%%
\subsection{Testes e validação do Classificador}
\label{subsec:testes_validacao}
\par{O classificador foi testado utilizando um conjunto de dados rigorosamente selecionado, composto por eventos rotulados como naturais por especialistas. Esta fase de testes foi crucial para validar a precisão do classificador na discriminação entre eventos antrópicos e naturais, ajustando parâmetros e refinando o modelo conforme necessário.}

\par{Os resultados dos testes foram encorajadores, mostrando uma boa capacidade do modelo em identificar corretamente a natureza dos eventos sismológicos. As métricas de desempenho, como precisão e recall, foram calculadas e apresentaram resultados satisfatórios, reforçando a eficácia do classificador desenvolvido.}

\par{Adicionalmente, foram realizados ajustes baseados nos resultados dos testes, incluindo a otimização da captura de dados e do pré-processamento, para melhorar a acurácia das classificações.}

%%%%%%%%%%%%%%%%%%%%%%%%%%%%%%%%%%%%%%%%%%%%%%%%%%%%%%%%%%%%%%%%%%%%%%%%%%%%%%%
\subsection{Implicações e futuras diretrizes}
\label{subsec:implicacoes}
\par{Os avanços alcançados com o desenvolvimento deste classificador sismológico abrem novas perspectivas para a análise de dados sismológicos no Brasil. Com a capacidade de discriminar de forma eficiente entre eventos naturais e antropogênicos, o classificador é uma ferramenta valiosa para o monitoramento ambiental e para a pesquisa geológica.}

\par{Recomenda-se a continuidade do desenvolvimento do sistema, com atualizações regulares do modelo e a integração de novas técnicas de análise de dados sismológicos. Além disso, é crucial manter a colaboração entre instituições de pesquisa nacionais e internacionais para aprimorar constantemente as capacidades de monitoramento sismológico do país.}

\par{A implementação do classificador em outras regiões e para diferentes tipos de dados sismológicos também será explorada, visando expandir sua aplicabilidade e contribuir para um conhecimento mais profundo da dinâmica sismológica regional e global.}

%%%%%%%%%%%%%%%%%%%%%%%%%%%%%%%%%%%%%%%%%%%%%%%%%%%%%%%%%%%%%%%%%%%%%%%%%%%%%%%
\subsection{Análise de Eventos em Horários Não Comerciais}
\label{subsec:nao_comerciais}


\begin{figure}[ht!]
	\captionsetup{justification=justified, singlelinecheck=false, width=1\textwidth}
    \caption{Mapa do Brasil mostrando pontos de interesse e os epicentros dos eventos classificados como detonações e sismos. Foram detectados um total de sessenta e sete (67) eventos associados a detonações no período, classificados a partir do horário de ocorrência e da forma de onda, além do plano de fogo fornecido, com magnitudes mínima e máxima de 0.4 e 3.0 MLv, respectivamente.}
    \begin{mdframed}[
        linecolor=black,
        linewidth=1pt,
        roundcorner=10pt,
    ]
    \begin{center}
    \includegraphics[width=0.8\textwidth]{/home/ggrl/projetos/ClassificadorSismologico/arquivos/figuras/mapas/mapa.png}
    \end{center}
    \end{mdframed}
    \caption*{Fonte: IPT}
\end{figure}


\par{Uma parte significativa do estudo envolveu a análise de eventos sismológicos registrados em horários não comerciais, definidos como o período entre as 23:00 UTC e 11:00 UTC. Este intervalo foi escolhido considerando as diferenças de fuso horário entre as várias regiões do Brasil, que abrangem de -3 UTC a -5 UTC. A análise focou em identificar características distintivas dos eventos naturais e antropogênicos ocorridos neste período.}

\par{Os dados foram processados e visualizados usando uma série de scripts Python desenvolvidos para filtrar, analisar e plotar informações sismológicas detalhadas. Os gráficos resultantes, como distribuições de probabilidade natural, boxplots de distância por natureza do evento e matrizes de correlação, ajudaram a ilustrar diferenças significativas nas características dos eventos registrados durante o horário não comercial.}

\par{Especificamente, a distribuição de \textit{prob\_nat} (probabilidade de um evento ser natural) para eventos naturais mostrou-se distinta daquela para eventos antropogênicos, com eventos naturais tendendo a ter valores mais altos de probabilidade. Além disso, a análise de recall por número de estações envolvidas no registro dos eventos revelou que um maior número de estações frequentemente correlaciona-se a uma classificação mais precisa entre eventos naturais e antropogênicos.}

\par{Os histogramas de recall ajustados para a hora do dia destacaram a precisão da classificação durante os horários não comerciais, refletindo a eficácia do algoritmo em identificar corretamente a natureza dos eventos sob condições variáveis de ruído ambiental e atividade humana.}

\par{Todas essas análises são cruciais para entender a dinâmica sismológica em horários menos típicos para atividades humanas, oferecendo insights sobre a influência de fatores naturais isolados das interferências antrópicas. Os resultados estão detalhados nos Apêndices A e B, onde figuras e tabelas fornecem uma representação visual e quantitativa das descobertas.}

\par{A validação destes resultados foi facilitada pela utilização de scripts automatizados que permitiram uma reprodutibilidade eficiente e a geração de outputs consistentes para análises subsequentes e revisões de estudo.}

\par{Este foco em eventos durante períodos não comerciais não apenas enriquece a compreensão da sismicidade natural mas também aprimora as metodologias de monitoramento e análise sismológica em condições controladas de ruído.}


                    \begin{figure}[H]
                        \centering
                        \includegraphics[width=1.0\textwidth]{/home/ipt/projetos/ClassificadorSismologico/arquivos/figuras/pos_processo/hist_ev_hour_recall.png}
                        \caption{Hist ev hour recall}
                        \label{fig:hist_ev_hour_recall}
                    \end{figure}
                

                    \begin{figure}[H]
                        \centering
                        \includegraphics[width=1.0\textwidth]{/home/ipt/projetos/ClassificadorSismologico/arquivos/figuras/pos_processo/dist_mean_snrs_recall.png}
                        \caption{Dist mean snrs recall}
                        \label{fig:dist_mean_snrs_recall}
                    \end{figure}
                

                    \begin{figure}[H]
                        \centering
                        \includegraphics[width=1.0\textwidth]{/home/ipt/projetos/ClassificadorSismologico/arquivos/figuras/pos_processo/dist_snrs.png}
                        \caption{Dist snrs}
                        \label{fig:dist_snrs}
                    \end{figure}
                

                    \begin{figure}[H]
                        \centering
                        \includegraphics[width=1.0\textwidth]{/home/ipt/projetos/ClassificadorSismologico/arquivos/figuras/pos_processo/boxplot_IM.png}
                        \caption{Boxplot im}
                        \label{fig:boxplot_IM}
                    \end{figure}
                

                    \begin{figure}[H]
                        \centering
                        \includegraphics[width=1.0\textwidth]{/home/ipt/projetos/ClassificadorSismologico/arquivos/figuras/pos_processo/dist_ev_distance_rel_freq.png}
                        \caption{Dist ev distance rel freq}
                        \label{fig:dist_ev_distance_rel_freq}
                    \end{figure}
                

                    \begin{figure}[H]
                        \centering
                        \includegraphics[width=1.0\textwidth]{/home/ipt/projetos/ClassificadorSismologico/arquivos/figuras/pos_processo/boxplot_BX.png}
                        \caption{Boxplot bx}
                        \label{fig:boxplot_BX}
                    \end{figure}
                

                    \begin{figure}[H]
                        \centering
                        \includegraphics[width=1.0\textwidth]{/home/ipt/projetos/ClassificadorSismologico/arquivos/figuras/pos_processo/boxplot_BL.png}
                        \caption{Boxplot bl}
                        \label{fig:boxplot_BL}
                    \end{figure}
                

                    \begin{figure}[H]
                        \centering
                        \includegraphics[width=1.0\textwidth]{/home/ipt/projetos/ClassificadorSismologico/arquivos/figuras/pos_processo/hist_mean_snrs_recall.png}
                        \caption{Hist mean snrs recall}
                        \label{fig:hist_mean_snrs_recall}
                    \end{figure}
                

                    \begin{figure}[H]
                        \centering
                        \includegraphics[width=1.0\textwidth]{/home/ipt/projetos/ClassificadorSismologico/arquivos/figuras/pos_processo/dist_prob_nat_recall.png}
                        \caption{Dist prob nat recall}
                        \label{fig:dist_prob_nat_recall}
                    \end{figure}
                

                    \begin{figure}[H]
                        \centering
                        \includegraphics[width=1.0\textwidth]{/home/ipt/projetos/ClassificadorSismologico/arquivos/figuras/pos_processo/dist_ev_cat_mag.png}
                        \caption{Dist ev cat mag}
                        \label{fig:dist_ev_cat_mag}
                    \end{figure}
                

                    \begin{figure}[H]
                        \centering
                        \includegraphics[width=1.0\textwidth]{/home/ipt/projetos/ClassificadorSismologico/arquivos/figuras/pos_processo/boxplot_rede.png}
                        \caption{Boxplot rede}
                        \label{fig:boxplot_rede}
                    \end{figure}
                

                    \begin{figure}[H]
                        \centering
                        \includegraphics[width=1.0\textwidth]{/home/ipt/projetos/ClassificadorSismologico/arquivos/figuras/pos_processo/dist_snrs_recall.png}
                        \caption{Dist snrs recall}
                        \label{fig:dist_snrs_recall}
                    \end{figure}
                

                    \begin{figure}[H]
                        \centering
                        \includegraphics[width=1.0\textwidth]{/home/ipt/projetos/ClassificadorSismologico/arquivos/figuras/pos_processo/boxplot_XV.png}
                        \caption{Boxplot xv}
                        \label{fig:boxplot_XV}
                    \end{figure}
                

                    \begin{figure}[H]
                        \centering
                        \includegraphics[width=1.0\textwidth]{/home/ipt/projetos/ClassificadorSismologico/arquivos/figuras/pos_processo/dist_mean_snrs_prob_nat.png}
                        \caption{Dist mean snrs prob nat}
                        \label{fig:dist_mean_snrs_prob_nat}
                    \end{figure}
                

                    \begin{figure}[H]
                        \centering
                        \includegraphics[width=1.0\textwidth]{/home/ipt/projetos/ClassificadorSismologico/arquivos/figuras/pos_processo/hist_ev_distance.png}
                        \caption{Hist ev distance}
                        \label{fig:hist_ev_distance}
                    \end{figure}
                

                    \begin{figure}[H]
                        \centering
                        \includegraphics[width=1.0\textwidth]{/home/ipt/projetos/ClassificadorSismologico/arquivos/figuras/pos_processo/boxplot_VL.png}
                        \caption{Boxplot vl}
                        \label{fig:boxplot_VL}
                    \end{figure}
                

                    \begin{figure}[H]
                        \centering
                        \includegraphics[width=1.0\textwidth]{/home/ipt/projetos/ClassificadorSismologico/arquivos/figuras/pos_processo/dist_mean_snrs.png}
                        \caption{Dist mean snrs}
                        \label{fig:dist_mean_snrs}
                    \end{figure}
                

                    \begin{figure}[H]
                        \centering
                        \includegraphics[width=1.0\textwidth]{/home/ipt/projetos/ClassificadorSismologico/arquivos/figuras/pos_processo/hist_mean_snrs.png}
                        \caption{Hist mean snrs}
                        \label{fig:hist_mean_snrs}
                    \end{figure}
                

                    \begin{figure}[H]
                        \centering
                        \includegraphics[width=1.0\textwidth]{/home/ipt/projetos/ClassificadorSismologico/arquivos/figuras/pos_processo/dist_ev_num_stations_absoluto.png}
                        \caption{Dist ev num stations absoluto}
                        \label{fig:dist_ev_num_stations_absoluto}
                    \end{figure}
                

                    \begin{figure}[H]
                        \centering
                        \includegraphics[width=1.0\textwidth]{/home/ipt/projetos/ClassificadorSismologico/arquivos/figuras/pos_processo/mean_snrs_3_prob_nat.png}
                        \caption{Mean snrs 3 prob nat}
                        \label{fig:mean_snrs_3_prob_nat}
                    \end{figure}
                

                    \begin{figure}[H]
                        \centering
                        \includegraphics[width=1.0\textwidth]{/home/ipt/projetos/ClassificadorSismologico/arquivos/figuras/pos_processo/dist_ev_cat_mag_recall.png}
                        \caption{Dist ev cat mag recall}
                        \label{fig:dist_ev_cat_mag_recall}
                    \end{figure}
                

                    \begin{figure}[H]
                        \centering
                        \includegraphics[width=1.0\textwidth]{/home/ipt/projetos/ClassificadorSismologico/arquivos/figuras/pos_processo/dist_ev_num_stations_recall.png}
                        \caption{Dist ev num stations recall}
                        \label{fig:dist_ev_num_stations_recall}
                    \end{figure}
                

                    \begin{figure}[H]
                        \centering
                        \includegraphics[width=1.0\textwidth]{/home/ipt/projetos/ClassificadorSismologico/arquivos/figuras/pos_processo/boxplot_dist.png}
                        \caption{Boxplot dist}
                        \label{fig:boxplot_dist}
                    \end{figure}
                

                    \begin{figure}[H]
                        \centering
                        \includegraphics[width=1.0\textwidth]{/home/ipt/projetos/ClassificadorSismologico/arquivos/figuras/pos_processo/hist_snrs_recall.png}
                        \caption{Hist snrs recall}
                        \label{fig:hist_snrs_recall}
                    \end{figure}
                

                    \begin{figure}[H]
                        \centering
                        \includegraphics[width=1.0\textwidth]{/home/ipt/projetos/ClassificadorSismologico/arquivos/figuras/pos_processo/region_corr.png}
                        \caption{Region corr}
                        \label{fig:region_corr}
                    \end{figure}
                

                    \begin{figure}[H]
                        \centering
                        \includegraphics[width=1.0\textwidth]{/home/ipt/projetos/ClassificadorSismologico/arquivos/figuras/pos_processo/boxplot_XC.png}
                        \caption{Boxplot xc}
                        \label{fig:boxplot_XC}
                    \end{figure}
                

                    \begin{figure}[H]
                        \centering
                        \includegraphics[width=1.0\textwidth]{/home/ipt/projetos/ClassificadorSismologico/arquivos/figuras/pos_processo/boxplot_ON.png}
                        \caption{Boxplot on}
                        \label{fig:boxplot_ON}
                    \end{figure}
                

                    \begin{figure}[H]
                        \centering
                        \includegraphics[width=1.0\textwidth]{/home/ipt/projetos/ClassificadorSismologico/arquivos/figuras/pos_processo/boxplot_Y4.png}
                        \caption{Boxplot y4}
                        \label{fig:boxplot_Y4}
                    \end{figure}
                

                    \begin{figure}[H]
                        \centering
                        \includegraphics[width=1.0\textwidth]{/home/ipt/projetos/ClassificadorSismologico/arquivos/figuras/pos_processo/mean_snrs_1_prob_nat.png}
                        \caption{Mean snrs 1 prob nat}
                        \label{fig:mean_snrs_1_prob_nat}
                    \end{figure}
                

                    \begin{figure}[H]
                        \centering
                        \includegraphics[width=1.0\textwidth]{/home/ipt/projetos/ClassificadorSismologico/arquivos/figuras/pos_processo/boxplot_BR.png}
                        \caption{Boxplot br}
                        \label{fig:boxplot_BR}
                    \end{figure}
                

                    \begin{figure}[H]
                        \centering
                        \includegraphics[width=1.0\textwidth]{/home/ipt/projetos/ClassificadorSismologico/arquivos/figuras/pos_processo/boxplot_NB.png}
                        \caption{Boxplot nb}
                        \label{fig:boxplot_NB}
                    \end{figure}
                

                    \begin{figure}[H]
                        \centering
                        \includegraphics[width=1.0\textwidth]{/home/ipt/projetos/ClassificadorSismologico/arquivos/figuras/pos_processo/hist_hora.png}
                        \caption{Hist hora}
                        \label{fig:hist_hora}
                    \end{figure}
                

                    \begin{figure}[H]
                        \centering
                        \includegraphics[width=1.0\textwidth]{/home/ipt/projetos/ClassificadorSismologico/arquivos/figuras/pos_processo/boxplot_GT.png}
                        \caption{Boxplot gt}
                        \label{fig:boxplot_GT}
                    \end{figure}
                

%%%%%%%%%%%%%%%%%%%%%%%%%%%%%%%%%%%%%%%%%%%%%%%%%%%%%%%%%%%%%%%%%%%%%%%%%%%%%%%
\subsection{Documentação e divulgação}
\label{subsec:documentacao}
\par{Todos os códigos e algoritmos desenvolvidos estão devidamente documentados e disponíveis publicamente no repositório do projeto. A documentação inclui guias de instalação, configuração e utilização do sistema, permitindo que outros pesquisadores e técnicos possam utilizar e adaptar o classificador para suas necessidades específicas.}

\par{Os resultados obtidos e as metodologias empregadas foram submetidos para publicação em periódicos especializados e apresentados em conferências nacionais e internacionais, contribuindo para a disseminação do conhecimento e das inovações desenvolvidas no âmbito deste projeto.}

\par{Continuará sendo dada ênfase à formação de parcerias estratégicas e ao engajamento da comunidade científica, visando fortalecer a rede de pesquisa em sismologia no Brasil e promover o uso de tecnologias avançadas na análise de fenômenos sismológicos.}



% CONSIDERAÇÕES FINAIS
\section{CONSIDERAÇÕES FINAIS}
\label{sec:consid_finais}

\par{Para o monitoramento sismológico realizado no período de 01.12.2022 a 30.06.2023, tem-se que:}

\begin{itemize}
    \item Em síntese, o funcionamento da Estação SP7 pôde ser considerado satisfatório. Um detalhamento maior do funcionamento da estação para o período englobado por este relatório pode ser obtido analisando-se os boletins sísmicos no Anexo A, que contêm os gráficos de completeza diários para cada mês no período.
    \item Na área de influência do empreendimento foram registrados sessenta e sete (67) desmontes com magnitudes entre 0.4 e 3.0 (MLv), sendo quatorze (19) destes relacionados a detonações nas proximidades das pedreiras Azza e Daclande, e os restantes em outras áreas (Figura 2, Apêndice A).
    \item Foram detectados três pequenos sismos locais próximos à estação SP7, com magnitudes entre 0.2 e 0.9 MLv. O maior destes pequenos eventos foi registrado em 2023-05-26 18:31:02 (UTC).
    \item Foi detectado um sismo regional natural no território brasileiro, próximo à cidade de Iguape – SP, em 2023-06-16 11:22:00 (UTC). O sismo teve magnitude 4.0 mR .
    \item Durante o monitoramento sismológico local efetuado com a Estação SP7 não foram registrados sismos induzidos oriundos da operação do reservatório. 
    \item A orientação e procedimentos apresentados no Relatório IPT no 115 463-205 – “Análise dos registros obtidos entre 1º de junho e 30 de novembro de 2009, na Estação Sismológica SP7, SC”, emitido em janeiro de 2010, e no Relatório IPT no 120 081-205 – “Análise dos registros obtidos entre 1º de junho e 30 de novembro de 2010 na Estação Sismológica SP7, Salto Pilão, SC e síntese das atividades e dos resultados do monitoramento sismológico”, emitido em janeiro de 2011, quanto à ocorrência de provável tremor de terra sentido pela população local ou ocorrências anômalas na área do empreendimento, devem ser mantidos.
\end{itemize}

\par{O monitoramento instrumental possibilita determinar o epicentro, quantificar o tamanho (a magnitude), definir a origem do evento e, se for o caso, em função da análise do comportamento espaço-temporal da atividade, tomar medidas mitigatórias.} 
\par{Assim, pelos resultados do monitoramento sismológico realizado com a Estação SP7, em função das características operacionais do registrador-sismômetro e de sua localização, considera-se de muita valia manter esta estação em funcionamento para dar continuidade no conhecimento da sismicidade local e regional.}
\par{No monitoramento realizado não foi observada a ocorrência de evento sísmico associado à implementação do Empreendimento da UHE Salto Pilão. A continuidade do monitoramento sismológico na área do empreendimento permitirá acompanhar eventual ocorrência local.}
\par{Mantidas as atuais características da sismicidade, local e regional, para a continuidade do monitoramento sismológico devem ser mantidos os atuais procedimentos adotados para operação, coleta e análise dos dados, com periodicidade mensal. Além disso, continuam válidas as orientações a serem adotadas no caso de ocorrer evento local sentido pela população e de eventuais anomalias na obra tais como desplacamentos de rochas e estampidos associados.}

\assinaturaDoisDigitalRelatorio



\clearpage
\newpage


% Bibliografia
\section{REFERÊNCIAS BIBLIOGRÁFICAS}
%\renewcommand{\refname}{REFERÊNCIAS}
%\addcontentsline{toc}{section}{REFERÊNCIAS}
\bibliography{ref}

\end{document}
