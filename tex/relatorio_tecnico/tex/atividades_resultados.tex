\section{ATIVIDADES REALIZADAS E RESULTADOS OBTIDOS}
\label{sec:ativ_real}
\par{O método de análise utilizado baseia-se nas técnicas atualmente empregadas pela Sismologia, desenvolvidas para a análise dos dados obtidos por meio de estação sismológica digital triaxial, com a utilização de softwares específicos.}

\par{Os resultados das análises dos dados registrados foram dispostos em figuras e tabelas que constam nos Apêndices A e B, respectivamente, para melhor apresentação e facilidade de leitura do texto deste Relatório.}

\subsection{Características operacionais}
\label{subsec:caracteristicas}
\par{Os dados foram obtidos por meio da Estação SP7, composta dos seguintes equipamentos:}
\begin{itemize}
    \item Digitalizador (DAS – Data Acquisition Sub-system), modelo 130-01 (RefTek, EUA), de alta resolução, com 3 canais de registro, 24 bits, número de série 9FD3;
    \item Sistema interno de gravação (DRS – Data Recording Sub-system) em flash cards (SimpleTech, EUA), capacidade de 2 GB (com 4 unidades, sendo duas instaladas na estação e, as outras duas, para troca na coleta de dados);
    \item Relógio GPS (RefTek);
    \item Sismômetro, modelo SeisMonitor (Oyo GeoSpace, EUA), número de série 1430;
    \item Painel solar, modelo SQ75-P de 75W/12Vcc, com controlador de carga BaseD (Artesa, Espanha), números de série 028222A1180544787 e 02090579, respectivamente;
    \item Bateria, estacionária, selada, 12 Vcc e 100 Ah (Freedom, DF2000);
    \item Palmtop, modelo MEL1000 (Meazura-Acceca, Austrália), utilizado para processar o software de operação dos equipamentos sismológicos. \\
A operação da estação é realizada por meio do software PFC-130 (Palm Field Controller), que é o controlador de ajuste de campo para o DAS130, versão palmtop. Com este software, são fornecidos, entre outros, os parâmetros de operação dos equipamentos, verificado o funcionamento da estação e transferidos os dados; e
    \item Multiteste, modelo ET1610 (Minipa, Brasil).
\end{itemize}

\par{Para transferência dos dados de campo para a sede do IPT, em São Paulo, foi utilizada a leitora de flash cards, modelo FlashLink (SimpleTech, EUA), número de série STI-UMC2300 e para arquivamento, um PC de uso geral do CESAP.}
\par{As coletas dos dados foram efetuadas por técnicos da UHSP, mediante troca dos flash cards. O encaminhamento dos dados para o IPT ocorreu a partir do envio dos mesmos na plataforma OneDrive para posterior processamento e análise.}
\par{A rotina de coleta e análise dos dados foi mantida. O Palmtop que serve para controle dos equipamentos da estação apresentou problemas de conexão na coleta do dia 14.06.2021, e consequentemente perdeu o arquivo de configurações da estação SP7 na memória. Após tentativas infrutíferas de restaurar a conexão com a estação, o Palmtop foi enviado ao IPT para reconfiguração, e foi consertado e enviado de volta à estação SP7, que resumiu sua operação normal no dia 01.09.2021. Desde então a estação opera normalmente. Entretanto, para o período entre 15.06.2021 e 01.09.2021 não houve registro de dados, e consequentemente desmontes/eventos sísmicos que pudessem ter ocorrido durante esse período não foram detectados. Desde então, a estação operou de modo satisfatório, sem que houvesse perdas de dados que comprometessem as análises.}
\par{Uma planilha de campo, contendo a sequência de tarefas que devem ser executadas a cada coleta de dados ou visita à estação, foi preenchida com informações que permitiram a avaliação do desempenho dos equipamentos instalados na estação sismológica. Além destas informações operacionais, foram anotados os problemas observados e toda atividade desenvolvida durante os trabalhos de campo. No Anexo A estão apresentadas as planilhas de campo.}
\par{O IPT realizou visita de campo à estação SP7 no dia 21.06.2022, para inspeção geral da estação sismológica e para instalação do novo equipamento GPS adquirido. A estação se encontra em bom estado de conservação, e o equipamento novo foi instalado sem grandes dificuldades, funcionando de imediato com o DAS em operação no local e adquirindo lock com a constelação de satélites em poucos segundos. Foi também averiguado que a provável fonte do ruído que se observa na estação de maneira intermitente em período comercial vem de uma bomba de água que se encontra instalada a aproximadamente 15 metros da estação. A estação opera de maneira satisfatória na maior parte do tempo, não comprometendo o monitoramento sismológico da área, e, portanto, este ruído intermitente é tolerável.}
\par{Além da verificação das informações contidas na planilha de campo e nos arquivos de gerenciamento do sistema, o técnico que está realizando a análise dos dados sismológicos faz uma interpretação das características dos sinais sísmicos para verificar se não está ocorrendo alguma anormalidade.}
\par{No dia 12.12.2007, às 19 h e 11 min (hora universal), iniciou-se a operação da Estação Sismológica SP7 situada nas coordenadas geográficas: 27,1203º S (6.999.291 UTM-N), 49,4620º W (652.443 UTM-E) e cota de 489 m (esta localização foi apresentada com detalhe no Desenho 1 do Relatório IPT no 98 859-205 – “Definição e instalação da estação sismológica na área do Aproveitamento Hidrelétrico Salto Pilão, SC”, emitido em fevereiro de 2008.}
\par{O sismômetro foi instalado e orientado de acordo com os pontos cardeais, de modo que os canais 1, 2 e 3 de registro correspondessem, respectivamente, às componentes Z (vertical), NS (Norte-Sul) e EW (Leste-Oeste). Vale observar que o cabo de ligação digitalizador-sismômetro veio montado com as polaridades das componentes horizontais invertidas. Devido à montagem deste cabo não foi possível inverter as ligações dos pinos destas componentes. Assim, foi decidido que, no IPT, quando processados os dados de campo, a inversão de polaridade seria realizada por meio do software de análise.}
\par{Na instalação da Estação SP7, foram utilizados os flash cards identificados com os números 1 e 2. Na primeira coleta de dados, estes flash cards foram substituídos pelos de números 3 e 4. Nas coletas seguintes, continuou-se a substituição, utilizando estes pares de flash cards de modo alternado.}
\par{O funcionamento da Estação SP7 pode ser visualizado, de modo geral, na Figura 1, Apêndice A. Nesta figura apresenta-se o gráfico de completeza mensal, mostrando a disponibilidade dos dados da estação SP7 no período de referência deste relatório. Em síntese, no período de monitoramento sismológico compreendido entre 01.12.2022 e 30.06.2023, o funcionamento da Estação SP7 pôde ser considerado satisfatório. Um detalhamento maior do funcionamento da estação para o período englobado por este relatório pode ser obtido analisando-se os boletins sísmicos no Anexo A, que contêm os gráficos de completeza diários para cada mês entre 01.12.2022 e 30.06.2023.}

\subsection{Interpretação dos registros}
\label{subsec:interpret}
\par{Este item abrange a interpretação de cada evento identificado, localizado e quantificado.}
\par{Para a determinação da distância epicentral, o azimute e a magnitude dos eventos registrados na Estação SP7, adotou-se a metodologia apresentada no Relatório IPT no 104 490-205 – “Análise dos registros obtidos entre 12 de dezembro de 2007 e 29 de maio de 2008, na Estação Sismológica SP7, SC”, emitido em julho de 2008.}

\subsubsection{Detonações em pedreiras e obras}
\par{O conhecimento e o controle do cronograma de detonações em pedreiras e obras situadas próximas à área do empreendimento são de grande importância na interpretação dos registros e análise dos dados, pois facilitam a identificação dos eventos associados às detonações.}
\par{O controle, de um modo geral, é efetuado por meio de planilhas, nas quais são declarados os dias e horários dos fogos, cargas, tipo de fogo e outras informações referentes às detonações efetuadas, que são periodicamente (geralmente mensalmente) enviadas ao IPT por mensagem eletrônica.}
\par{Na área do entorno do empreendimento foram identificadas duas pedreiras em funcionamento (pedreiras Azza e Daclande) e vários pontos com atividades minerárias que podem utilizar explosivos no processo de lavra. Destaca-se que no Desenho 1 do Relatório IPT no 98 859‑205 foram apresentadas as localizações destas pedreiras e minerações.}
\par{Neste período de análise foram recebidas informações sobre detonações das pedreiras Azza e Daclande para os meses de dezembro de 2022 e janeiro, fevereiro, março, abril, maio e junho de 2023. Com os dados das detonações realizadas procurou-se identificá-las nos registros dos eventos obtidos instrumentalmente. Os eventos foram classificados como detonações suspeitas ou sismos a partir da análise da forma de onda do registro sísmico e do horário de ocorrência, junto à inspeção via satélite pelo software Google Earth Pro, para averiguar a presença de operações de mineração nas proximidades dos epicentros determinados.}
\par{Foi identificado um total de sessenta e sete (67) desmontes/detonações no período, sendo sete (7) destes confirmados através do plano de fogo provido pelas pedreiras (Daclande e Azza), e outros doze (12) localizados nas imediações das mesmas, as quais foram relacionadas a detonações não informadas por estas pedreiras devido à localização dos epicentros, às características dos sinais sísmicos e aos horários de ocorrência semelhantes aos das detonações que já vinham ocorrendo nestes locais. Já os outros quarenta e oito (48) eventos registrados não puderam ser correlacionados com os dados recebidos das pedreiras conhecidas, mas pela característica do sinal sísmico, horário de ocorrência e pela verificação de que existem empreendimentos de mineração/pedreiras não identificadas na região, foram classificados como tal.}
\par{Ressalta-se que todos os registros classificados como detonações (com ou sem confirmação) apresentaram magnitude pequena, condizente com diferentes tipos de detonações nestes empreendimentos, sendo as magnitudes mínima e máxima registradas de 0.4 e 3.0 MLv, respectivamente.}
\par{A Tabela 1, Apêndice B, contém a relação dos eventos que foram associados às detonações, incluindo a data e o horário de registro (em hora universal), o epicentro, magnitude e energia liberada, além de um comentário indicando a pedreira associada, quando o epicentro localizado consta no plano de fogo. A Figura 2, Apêndice A, mostra a localização em volta do empreendimento para as detonações identificadas no período.}
\par{Como as informações advindas de detonações sempre serão úteis na interpretação dos dados, solicita-se que, periodicamente, seja verificada a existência de novas pedreiras/obras na área, obtidas as suas respectivas localizações e efetuados os contatos com os responsáveis, com o intuito de se obter os correspondentes “planos de fogo”.}

\subsubsection{Sismos naturais}
\par{No período de 01.12.2022 a 30.06.2023, foram detectados três pequenos sismos locais próximos à estação SP7, com magnitudes entre 0.2 e 0.9 MLv. O maior destes eventos foi registrado em 2023-05-26 18:31:02 (UTC). Suas localizações com relação à estação SP7 são mostradas na Figura 2, Apêndice A.}
\par{Foi detectado um sismo regional natural no território brasileiro, na região do Vale do Ribeira, no estado de São Paulo, próximo à cidade de Iguape – SP, em 2023-06-16 11:22:00 (UTC). O sismo teve magnitude 4.0 mR, e foi sentido por diversas pessoas na região e também em áreas distantes do epicentro, como na capital paulista. O registro do evento na estação SP7 é mostrado na Figura 3, Apêndice A. O registro pela estação teve excelente qualidade e auxiliou a determinação do mecanismo focal (determinação de orientação da falha) do evento em conjunto com as estações da Rede Sismográfica Brasileira.}
\par{Não foram detectados telessismos no território brasileiro durante o período. Os telessismos fora do Brasil e imediações, embora registrados, não foram avaliados neste estudo que visa caracterizar a sismicidade ocorrida na área da Usina Salto Pilão. Os epicentros localizados no Brasil são localizados para contribuir com o melhor conhecimento da sismicidade nacional.}
\par{Continuam válidas as considerações e orientações anteriores a respeito das medidas a serem tomadas em caso de ocorrência de um sismo local sentido pela população.}
\par{Para o período de 01.12.2022 a 30.06.2023 foram emitidos os Boletins Sísmicos nos 18/36-2024 a 24/36-2024, apresentados no Anexo A.}

\subsubsection{Sismos induzidos}
\par{O monitoramento da sismicidade induzida por reservatório na área de influência do Empreendimento da Usina Hidrelétrica Salto Pilão, que está na fase do pós-enchimento, não indicou a existência de eventos relacionados à operação deste reservatório.}
\par{Porém, recomenda-se que a auscultação sismológica na região seja mantida, de modo a dar continuidade ao melhor conhecimento da sismicidade local e regional, além do necessário acompanhamento da operação da UHE Salto Pilão.}

