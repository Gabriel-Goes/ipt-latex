% Parâmetros
\docNum{XXX-XXX}
\tipo{RELATÓRIO TÉCNICO}
\cancelaDoc{Relatório Técnico nº 123 456-205} % Descomente para opção cancela e substitui
\cliente{Consórcio empresarial Salto Pilão}{CESAP}{}{}{}{}
\data{10 de outubro de 2023}
\titulo{Análise dos registros obtidos entre na estação Sismológica SP7, Salto Pilão, SC.}
\unidade{Cidades Infraestruturas e Meio Ambiente}{CIMA}
\lab{Seção de Obras Civis - SOC}
\periodo{01 de dezembro de 2022}{30 de junho de 2023}

% Inserir capa
\capa

% Inserir Resumo e Tabela de Conteúdo
\pagestyle{tipo_pre}
\resumo{Neste Relatório são apresentados os resultados do monitoramento sismológico efetuado na área da Usina Hidrelétrica Salto Pilão, por meio da Estação Sismológica SP7, no período entre 01.12.2022 e 30.06.2023, permitindo acompanhar a sismicidade local e orientar a adoção de eventuais medidas mitigadoras. Durante o monitoramento sismológico local efetuado, a Estação SP7 registrou sessenta e sete (67) desmontes em obras/pedreiras na região. No período de referência do presente relatório não foi observada a ocorrência de evento sísmico induzido pela implementação do Empreendimento da UHE Salto Pilão. Foram detectados 4 sismos naturais, sendo 3 destes sismos locais próximos à estação SP7, e um evento regional próximo à cidade de Iguape – SP, no estado de São Paulo em 2023-06-16 11:22:00 (UTC) com magnitude 4.0 mR. Ressalta-se a contribuição que este monitoramento sismológico está dando para a confirmação e a determinação dos parâmetros de eventos ocorridos no território brasileiro, em especial aqueles com epicentros nos estados de Santa Catarina e Rio Grande do Sul, e regiões vizinhas. Assim, recomenda-se que a Estação SP7 seja mantida em funcionamento, possibilitando dar continuidade ao melhor conhecimento da sismicidade local e regional, além do necessário acompanhamento da operação da Usina Hidrelétrica Salto Pilão.}{Sismologia; sismicidade; Salto Pilão; sismos induzidos; sismos naturais; detonações; UHE Salto Pilão.}
\tableofcontents

\pagebreak
\renewcommand{\thepage}{\arabic{page}}
