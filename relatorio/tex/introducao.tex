\section{Introdução}
\par{Este Relatório integra o estudo sismológico em desenvolvimento pelo Consórcio Empresarial Salto Pilão - CESAP e o IPT, de acordo com a Proposta FIPT/IPT n° 59220/21 de 29 de junho de 2021 - “Monitoramento sismológico na área do AHE Salto Pilão, SC, entre julho/2021 e junho/2024”, em continuidade aos trabalhos iniciados em janeiro de 2007 (Carta Proposta CT-Obras/SG-233/06), referentes à implantação do programa de monitoramento da sismicidade induzida na Usina Hidrelétrica Salto Pilão - UHESP. O estudo visa o atendimento aos requisitos do Projeto Básico Ambiental - PBA deste empreendimento, em execução na bacia do rio Itajaí-Açu, nos municípios de Lontras, Ibirama e Apiúna, no Estado de Santa Catarina, dentro do Programa 3: Monitoramento dos Impactos Geológicos – Sub-Programa 3.2: Sismicidade Induzida, de acordo com a LAO – Licença Ambiental de Operação no 4.055/12 concedida pela Fundação do Meio Ambiente - FATMA do Estado de Santa Catarina, atualmente Instituto de Meio Ambiente de Santa Catarina - IMA.}

\subsection{Objetivo}
\par{O objetivo deste trabalho é apresentar os resultados do monitoramento sismológico efetuado na área da Usina Hidrelétrica Salto Pilão, com a Estação Sismológica SP7, entre 01 de dezembro de 2022 e 30 de junho de 2023, permitindo acompanhar a sismicidade local e orientar a adoção de eventuais medidas mitigadoras, atendendo às exigências previstas no processo de licenciamento ambiental do empreendimento.}
