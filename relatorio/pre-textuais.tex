\documentclass{article}
\usepackage{geometry}
\usepackage{setspace}
\usepackage{titlesec}
\usepackage{titletoc}
\usepackage{tocloft}
\usepackage{booktabs}
\usepackage{adjustbox}
\usepackage{hyperref}
\usepackage{fancyhdr}
\usepackage{lipsum}
\usepackage{romanbar}
\usepackage{etoolbox}
\usepackage{lastpage}
\usepackage{csvsimple}
\usepackage[bottom]{footmisc}
\usepackage{siunitx}
\sisetup{round-mode=places, round-precision=0}
\usepackage{numprint}

% Configurações de Fonte
\renewcommand{\rmdefault}{phv} % Arial
\renewcommand{\sfdefault}{phv} % Arial
\geometry{a4paper, left=2.5cm, right=2cm, top=4.8cm, bottom=3.5cm}
\renewcommand{\baselinestretch}{1.5}

% Configuração do espaçamento antes e depois de títulos
\titlespacing*{\section}{0pt}{1.5\baselineskip}{0.6\baselineskip}
\titlespacing*{\subsection}{0pt}{1.5\baselineskip}{0.6\baselineskip}
\titlespacing*{\subsubsection}{0pt}{1.5\baselineskip}{0.6\baselineskip}
\titlespacing*{\paragraph}{0pt}{1.5\baselineskip}{0.6\baselineskip}
\titlespacing*{\subparagraph}{0pt}{0pt}{0.6\baselineskip}

% Configuração de formatação de listas
\renewcommand{\labelitemi}{\textbullet}

\pagestyle{fancy}
\fancyhf{} % Limpa os cabeçalhos e rodapés
\fancyhead[L]{\vspace*{-0cm}\includegraphics[width=4cm]{Picture1.png}} % Imagem no cabeçalho
\fancyfoot[R]{\vspace*{-0cm}\includegraphics[width=8cm]{Picture2.png}} % Imagem no rodapé
\renewcommand{\headrulewidth}{0pt} % Remove a linha horizontal do cabeçalho
\renewcommand{\footrulewidth}{0pt} % Remove a linha horizontal do rodapé

\begin{document}
\begin{titlepage}
\title{INSTRUÇÃO NORMATIVA TC-10 \\ PROCEDIMENTO PARA A ELABORAÇÃO E EMISSÃO DE RELATÓRIO TÉCNICO}
\maketitle
\end{titlepage}

\pagenumbering{Roman}
\setcounter{page}{1}
\fancyhead[R]{\textbf{ Relatório Técnico nº XXX {\thepage}}} % Número de página no canto superior direito

% FOLHA DE ROSTO
\section*{FOLHA DE ROSTO}
Aqui deve ser adicionada uma folha de rosto modelo para todos os arquivos. Formulário 1153 [disponível na intranet]\url{http://portal.ipt.br} > Institucional > Comunicação > Logotipos e Modelos > Papeis Timbrados.
\newpage

% AGRADECIMENTOS
\section*{Agradecimentos}
Inúmeros órgãos públicos e entidades prestaram a sua colaboração, de alguma forma, no fornecimento de dados que constam deste relatório. Mesmo correndo o risco de alguma omissão, dado o grande número de consultas realizadas ao longo do desenvolvimento dos trabalhos, não se pode deixar de mencionar:
\begin{itemize}
    \item Departamento de Águas e Energia Elétrica - DAEE
    \item Companhia de Tecnologia de Saneamento Ambiental - Cetesp
    \item Companhia de Saneamento Básico do Estado de São Paulo - Sabesp
    \item Fundação Prefeito Faria Lima - Cepam
\end{itemize}
\newpage

% RESUMO
\section*{Resumo}
Condensação de no máximo 500 palavras em um único parágrafo, delineando e enfatizando resultados, conclusões e pontos mais relevantes do trabalho.
\newpage

% SUMÁRIO
\tableofcontents
\newpage

% LISTA DE ILUSTRAÇÕES
\listoffigures
\newpage

% LISTA DE TABELAS
\listoftables
\newpage

% LISTA DE SIGLAS, ABREVIATURAS, SÍMBOLOS E CONVENÇÕES
\section*{Lista de Siglas, Abreviaturas, Símbolos e Convenções}
% Conteúdo para a lista

% ELEMENTOS TEXTUAIS
\clearpage
\pagenumbering{arabic} % Começa a numeração com algarismos arábicos
\setcounter{page}{1}
\setcounter{section}{0} % Reinicia a contagem das seções
\fancyhead[R]{\textbf{ Relatório Técnico nº XXX {\thepage}/\pageref{LastPage}}} % Número de página no canto superior direito

% Configuração do estilo da página
\pagestyle{fancy}
\fancyhead[L]{\vspace*{-0cm}\includegraphics[width=4cm]{Picture1.png}} % Imagem no cabeçalho
\fancyfoot[R]{\vspace*{-0cm}\includegraphics[width=8cm]{Picture2.png}} % Imagem no rodapé
\renewcommand{\headrulewidth}{0pt} % Remove a linha horizontal do cabeçalho
\renewcommand{\footrulewidth}{0pt} % Remove a linha horizontal do rodapé


% INTRODUÇÃO
\section{INTRODUÇÃO}
Isto é apenas um texto de introdução onde eu enrolo um pouco para identificar como funciona um texto corrido em LaTeX. Será que parágrafos são feitos com TAB? Vamos ver a seguir.

Isto deveria ser um parágrafo. Gostaria que o ChatGPT corrigice isto para virar um parágrafo.

Ele não conseguiu, mas me ensinou como faz um.

\newpage

% DESENVOLVIMENTO
\section{DESENVOLVIMENTO}
Chato Gepeto criou esta tabela pra mim, mas não acredito muito nas criações dele. Será  que funciona?

\begin{center}
    \begin{table}[htbp]
        \caption{Dados de Terremotos}
        \label{tab:csv_example}
        \renewcommand{\arraystretch}{1.5} % Ajusta espaçamento entre linhas da tabela
        \small
        \begin{tabular}{ccccccccc} % Defina o número de colunas de acordo com seu arquivo CSV
            \toprule
            ID & Hora de Origem (UTC) & Longitude & Latitude &
            \multicolumn{1}{c}{UTM X} & \multicolumn{1}{c}{UTM Y} & MLv & Energia & Cat \\
            \midrule
            \csvreader[
                table head=\toprule \csvlinetotablerow \\ \midrule,
                late after line=\\, % Quebra de linha após cada linha do CSV
                late after last line=\\, % Quebra de linha após a última linha
            ]{events.csv}{}{% Substitua "dados.csv" pelo nome do seu arquivo CSV
                \csvcoli & \csvcolii & \csvcoliii & \csvcoliv &
            \num[group-separator={.},round-mode=places, round-precision=0]{\csvcolv} & \num[group-separator={.},round-mode=places, round-precision=0]{\csvcolvi} & \csvcolvii & \csvcolviii & \csvcolix
            }
            \bottomrule
        \end{tabular}
    \end{table}
\end{center}
\newpage

% CONSIDERAÇÕES FINAIS
\section{CONSIDERAÇÕES FINAIS}
\newpage

% INFORMAÇÕES COMPLEMENTARES
\section{INFORMAÇÕES COMPLEMENTARES}
\newpage


% EQUIPE TÉCNICA
\section{EQUIPE TÉCNICA}
\newpage


% LOCAL, DATA E ASSINATURAS
\section{LOCAL, DATA E ASSINATURAS}
\newpage


% ELEMENTOS PÓS-TEXTUAIS

% REFERÊNCIAS
\section{REFERÊNCIAS}
\newpage


% GLOSSÁRIO
\section{GLOSSÁRIO}
\newpage


% APÊNDICE
\section{APÊNDICE}
\newpage


% ANEXOS
\section{ANEXOS}
\newpage


% CONTRACAPA
\section{CONTRACAPA (emissão impressa, em papel)}
\newpage

\end{document}

