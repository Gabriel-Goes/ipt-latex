\documentclass[12pt]{article}
\usepackage[main=brazil]{babel}
\usepackage{geometry}
\geometry{a4paper, margin=2.5cm}
\usepackage{graphicx}
\usepackage{longtable,booktabs}
\usepackage{siunitx}
\sisetup{
    group-separator={.},
    group-minimum-digits=3,
    output-decimal-marker={,}}

\title{Relatório de Controle de Qualidade e Sincronização}
\author{Equipe Sismologia}
\date{\today}

\begin{document}

\maketitle
\tableofcontents
\clearpage

\section{Resumo Executivo}
Este relatório apresenta os resultados do controle de qualidade e sincronização dos dados sísmicos para os canais disponíveis no sistema. Os resultados estão detalhados nas seções seguintes, com foco nos períodos disponíveis, lacunas nos dados e operações realizadas.

\section{Dados de Controle de Qualidade}
\subsection{Canais Disponíveis}
\begin{itemize}
    \item MC\_MC9\_\_HHE
    \item MC\_MC9\_\_HHN
    \item MC\_MC9\_\_HHZ
\end{itemize}

\subsection{Períodos Disponíveis}
\begin{longtable}{@{}lll@{}}
\toprule
\textbf{Canal} & \textbf{Início} & \textbf{Fim} \\ \midrule
HHZ.D & 2024/296 & 2024/338 \\
HHE.D & 2024/296 & 2024/338 \\
HHN.D & 2024/296 & 2024/338 \\ \bottomrule
\end{longtable}

\subsection{Lacunas nos Dados}
\begin{itemize}
    \item \textbf{HHZ.D}: 3 segmentos, gaps de até 100.8 segundos.
    \item \textbf{HHE.D}: 3 segmentos, gaps de até 100.6 segundos.
    \item \textbf{HHN.D}: 3 segmentos, gaps de até 99.32 segundos.
\end{itemize}

\section{Sincronização}
\subsection{Resumo de Operações}
\begin{longtable}{@{}ll@{}}
\toprule
\textbf{Operação} & \textbf{Quantidade} \\ \midrule
Arquivos Processados & 1095 \\
Arquivos Copiados    & 126 \\
Arquivos Mesclados   & 3 \\
Arquivos Sobrepostos & 3 \\
Erros               & 0 \\ \bottomrule
\end{longtable}

\section{Referências Bibliográficas}
C. F. RICHTER, \textit{Elementary Seismology}, W. H. Freeman and Co., San Francisco, 1958, 768 pp.

\end{document}
