\pagestyle{geral}
% Corpo do documento
\section{ÚLTIMOS RELATÓRIOS TÉCNICOS}
\label{sec:ultimos_relatorios}
\begin{itemize}
    \item Relatório IPT Nº 169 991-205 - "Análise dos registros obtidos entre 01 de dezembro de 2022 e 30 de junho de 2023 na estação Sismológica SP7, Salto Pilão, SC.", emitido em agosto de 2023. 
\end{itemize}

\section{ATIVIDADES REALIZADAS}
\label{sec:atividade}
\begin{itemize}
    \item Encaminhamento do boletim sísmico nº 26/36-2024, Agosto-2023;
    \item Coleta de dados em 21/09/2023 e envio dos mesmos para análise no IPT;
    \item Processamento preliminar dos dados do período (01/08/2023 a 31/08/2023), referente às coletas SP723230 (05/07/2023 a 18/08/2023) e SP723264 (18/08/2023 a 21/09/2023); 
    \item Elaboração do gráfico de completeza dos dados, tabela contendo os registros de eventos/detonações detectados e mapa da região indicando os epicentros localizados;
	\item  Comparação dos desmontes detectados aos planos de fogo das pedreiras Azza e Daclande.
\end{itemize}

\section{RESULTADOS}
\label{sec:resultados}
\begin{itemize}
	\item A análise preliminar dos dados mostrou que a estação SP7 teve funcionamento normal e satisfatório durante o período de análise deste boletim sísmico (18/08/2023 a 21/09/2023). O gráfico de completeza para o período é mostrado na Figura 1. 
    \item Foram detectados quinze desmontes na área do entorno do empreendimento durante o período, sendo três na região das pedreiras Azza e Daclande (sem confirmação pelo plano de fogo das pedreiras) e os restantes localizados em outras pedreiras/áreas de mineração identificadas por imagens de satélite. O maior desmonte detectado no período teve magnitude 1.8 MLv e foi detectado em 2023-08-21 20:32:38 (UTC), longe da região do reservatório. O desmonte foi assim classificado por sua forma de onda sísmica, epicentro, horário de ocorrência e magnitude. 
    \item Durante o período não foram detectados sismos naturais locais, regionais e/ou telessismos em território brasileiro pela estação SP7. 
    \item Os parâmetros epicentrais dos eventos se encontram detalhados na Tabela 1. O mapa de entorno da região junto da localização dos epicentros é mostrado na Figura 2. 
\end{itemize}
    

\section{CONSIDERAÇÕES}
\label{sec:consideracoes}
Continuam válidas as considerações e orientações anteriores a respeito das medidas a serem tomadas em caso ocorrência de um sismo local sentido pela população, i.e., coletar os relatos da população local através de questionários macrossísmicos, contactar a defesa civil para avaliar possíveis danos em estruturas e fornecer orientações e informações à população. 

\assinaturaLucas
\clearpage
\newpage
