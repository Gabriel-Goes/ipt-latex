\documentclass[12pt]{iptex}
% IDIOMA
% Se necessário, inserir línguas e indicar língua principal (main)
\usepackage[main=brazil,english]{babel}
\usepackage[italic]{mathastext}
\sisetup{
    group-separator={.},
    group-minimum-digits=3,
    output-decimal-marker={,}}

% Alterando nome do TOC (adicionar em outras línguas, se necessário)
\addto\captionsbrazil{%
	\renewcommand{\contentsname}{SUMÁRIO}
	\renewcommand{\refname}{REFERÊNCIAS}
}

% BIBLIOGRAFIA
%\usepackage[abnt-emphasize=bf,alf]{abntex2cite}
% Para bibliografia ABNT com números
\usepackage[abnt-emphasize=bf,num]{abntex2cite}
\citebrackets[]

% Para outros estilos o autor deve definir o \bibliographystyle dentro do documento

% Início do documento
\begin{document}

% CAPA 
% Params
\tipo{BOLETIM SISMOLÓGICO}
\data{2023}
\titulo{\textbf{RSIS - Rede Sismológica Itá/Machadinho} \\
\textbf{Reservatório Salto Pilão, RS} \\
\textbf{BOLETIM SÍSMICO Nº XXXXXX}}
\unidade{Cidades Infraestruturas e Meio Ambiente}{CIMA}
\lab{Seção de Obras Civis - SOC}
\periodo{MES/ANO}{MES/ANO}

% Inserir capa
\capa
\pagestyle{timbrado}

\vspace{0.5cm}

\pagestyle{geral}


% Corpo
% Corpo do documento
\section{ÚLTIMOS RELATÓRIOS TÉCNICOS}
\label{sec:ultimos_relatorios}
\begin{itemize}
    \item Relatório Síntese UHMC 2023: Monitoramento sismológico na área do reservatório de Aproveitamento Hidrelétrico de Machadinho, SC/RS, emitido em abril de 2023.
    \item Relatório IPT Nº 205 166 666-1 - “Análise dos registros obtidos entre 01 de dezembro de 2019 e 31 de dezembro de 2021 na rede Sismológica de Itá/Machadinho, RSIM, SC/RS.”, emitido em novembro de 2022.
\end{itemize}

\section{ATIVIDADES REALIZADAS}
\label{sec:atividade}
\begin{itemize}
    \item Encaminhamento do Boletim sísmico nº 28/48-2024, Agosto-2023;
    \item Coleta de dados em 01/09/2023 (03/08/2023 a 01/09/2023) e envio dos mesmos para análise no IPT;
    \item Para o período, não houve acesso ao plano de fogo da obra PCH Tupitinga e das pedreiras Engenhos, Kerbermix e PlanaTerra;
    \item Análise preliminar do período que inclui as coletas BCM223215 (03/07/2023 a 03/08/2023) e BCM223244 (03/08/2023 a 01/09/2023); e
    \item Elaboração de gráfico de completeza dos dados, tabela contendo os registros de eventos/detonações detectados.
\end{itemize}

\section{RESULTADOS}
\label{sec:resultados}
Foi detectado um único sismo induzido na região do empreendimento de Machadinho durante o período, na região do remanso do reservatório, próximo à estação BCM2 com magnitude -0.9 MLv, evento pequeno, em 2023-08-09 10:53:18 (UTC). Não há relatos de eventos que tenham sido sentidos pela população local.

Foram detectados 14 (quatorze) desmontes durante o período, sendo o de maior magnitude em 2023-08-01 18:17:15 (UTC) com magnitude 1.9 MLv. Todos os desmontes ocorreram longe da região do reservatório (incluindo o de maior magnitude), com exceção de um desmonte ocorrido ao sul da cidade de Campos Novos – SC.

Foi detectado um sismo natural regional suspeito em 2023-08-15 19:15:41 (UTC) próximo ao distrito de Santa Terezinha do Salto - Lages/SC, com magnitude calculada 1.5 MLv. Não foram detectados telessismos no território brasileiro durante o período englobado por este boletim na estação BCM2.

Os parâmetros sísmicos dos eventos detectados são detalhados na Tabela 1. O gráfico de completeza dos dados para a estação BCM2 no mês de agosto/2023 é mostrado na Figura 1.

O funcionamento da estação BCM2 foi adequado no mês de agosto/2023. A estação MC9, que se encontrava avariada, conforme detalhado no boletim sísmico Nº 38/48-2021 Jul.20, teve seus equipamentos substituídos e foi reinstalada no dia 27/09/2023. O antigo digitalizador da estação foi retornado à sede do Consórcio Machadinho em Piratuba/SC. A estação voltou a funcionar corretamente e seus dados serão incluídos na próxima análise.

\section{CONSIDERAÇÕES}
\label{sec:consideracoes}
Continuam válidas as considerações e orientações anteriores a respeito das medidas a serem tomadas em caso ocorrência de um sismo local sentido pela população, i.e., coletar os relatos da população local através de questionários macrossísmicos, contactar a defesa civil para avaliar possíveis danos em estruturas e fornecer orientações e informações à população.

A estação MC9 voltou a operar normalmente com novos equipamentos em 27/09/2023, e seus dados serão incluídos na próxima análise.

\assinaturaLucas


\section{COMPLETUDE DOS DADOS}

\begin{figure}[htb!]
    \centering
	\captionsetup{justification=raggedright, singlelinecheck=false, width=1\textwidth}
    \caption{Gráfico de completude dos dados para o mês de MÊS para estação ESTAÇÃO.}
    \includegraphics[width=1.0\textwidth]{./boletim/main/figuras/completude.png} % Substitua pelo nome da imagem e ajuste o tamanho
    \caption*{Fonte: IPT}
    \label{fig:completude}
\end{figure}



\input{./boletim/main/tex/tabela_main_boletim.tex}


    \begin{figure}[hp]
    \centering
<<<<<<< HEAD
	\captionsetup{justification=justified, singlelinecheck=false, width=1\textwidth}
    \caption{Mapa da região de interesse no entorno do empreendimento, mostrando as principais cidades, rodovias e rios, com a localização das pedreiras, estação \textbf{SP7}, e eventos próximos ao empreendimento detectados no período de interesse.}
=======
    \captionsetup{justification=justified, singlelinecheck=false, width=1\textwidth}
    \caption{Mapa da região de interesse no entorno do empreendimento, mostrando as principais cidades, rodovias e rios, com a localização das pedreiras, estações \textbf{BCM2} e \textbf{MC9}, e eventos próximos ao empreendimento detectados no período de interesse.}
>>>>>>> 157db0ea11ef4bcaca122b47262699549e3e7111
    \includegraphics[width=1.0\textwidth]{./boletim/main/figuras/mapaevents.png}
    \caption*{Fonte: IPT}
    \end{figure}
    \newpage
    

% Bibliografia
\clearpage
\section{REFERÊNCIAS BIBLIOGRÁFICAS}

C. F. RICHTER, \textit{Elementary Seismology}, W. H. Freeman and Co., San Francisco, 1958, 768 pp.
%\renewcommand{\refname}{REFERÊNCIAS}
%\addcontentsline{toc}{section}{REFERÊNCIAS}
%\bibliography{ref}

\end{document}
