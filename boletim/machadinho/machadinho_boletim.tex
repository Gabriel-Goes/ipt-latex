\documentclass[12pt]{iptex}
% IDIOMA
% Se necessário, inserir línguas e indicar língua principal (main)
\usepackage[main=brazil,english]{babel}
\usepackage[italic]{mathastext}
\usepackage{siunitx}
\sisetup{
    group-separator={.},
    group-minimum-digits=3,
    output-decimal-marker={,}}

% Alterando nome do TOC (adicionar em outras línguas, se necessário)
\addto\captionsbrazil{%
	\renewcommand{\contentsname}{SUMÁRIO}
	\renewcommand{\refname}{REFERÊNCIAS}
}
\addto\captionsenglish{%
	\renewcommand{\contentsname}{CONTENTS}
	\renewcommand{\refname}{REFERENCES}
}
% BIBLIOGRAFIA
\usepackage[abnt-emphasize=bf,alf]{abntex2cite}
% Para bibliografia ABNT com números
%\usepackage[abnt-emphasize=bf,num]{abntex2cite}
%\citebrackets[]

% Para outros estilos o autor deve definir o \bibliographystyle dentro do documento

% Ajustes adicionais
%\setlist[enumerate]{itemsep=3mm} % espaçamento enumerate
%\setlist[itemize]{itemsep=3mm} % espaçamento itemize

% Início do documento
\begin{document}

% CAPA 
% Params
\tipo{BOLETIM SISMOLÓGICO}
\data{2023}

\titulo{Rede Sismológica de Itá/Machadinho - RSIM \\
          UHE Machadinho - SC/RS \\
          BOLETIM SÍSMICO Nº 29/36-2024 Out.23 }

\unidade{Cidades Infraestrutura e Meio Ambiente}{CIMA}
\lab{Seção de Obras Civis - SOC}
\periodo{01/09/2023}{30/09/2023}

% Inserir capa
\capa
\pagestyle{timbrado}

\vspace{0.5cm}

\pagestyle{geral}



% Corpo
\pagestyle{geral}
% Corpo do documento
\section{ÚLTIMOS RELATÓRIOS TÉCNICOS}
\label{sec:ultimos_relatorios}
\begin{itemize}
	\item "Relatório Síntese UHMC 2023: Monitoramento sismológico na área do reservatório de Aproveitamento Hidrelétrico de Machadinho, SC/RS.", emitido em abril de 2023. 
    \item Relatório IPT Nº 205 166 666-1 -- "Análise dos registros obtidos entre 01 de dezembro de 2019 e 31 de dezembro de 2021 na rede Sismológica de Itá/Machadinho, RSIM, SC/RS.", emitido em novembro de 2022.  
\end{itemize}

\section{ATIVIDADES REALIZADAS}
\label{sec:atividade}
\begin{itemize}
    \item Encaminhamento do boletim sísmico nº 29/36-2024, setembro-2023;
    \item Coleta de dados em 03/10/2023 e envio dos mesmos para análise no IPT;
    \item Processamento preliminar dos dados do período (01/09/2023 a 30/09/2023);
    \item Para o período, não houve acesso ao plano de fogo da obra PCH Tupitinga e das pedreiras Kerbermix e PlanaTerra. A Pedreira Engenhos forneceu plano de fogo;
    \item Análise preliminar do período que inclui as coletas BCM223244 (03/08/2023 a 01/09/2023), BCM223276 (01/09/2023 a 03/10/2023) e MC923276 (26/09/2023 a 03/10/2023); e   
    \item Elaboração do gráfico de completeza dos dados, tabela contendo os registros de eventos/detonações detectados e mapa da região indicando os epicentros localizados;
\end{itemize}

\section{RESULTADOS}
\label{sec:resultados}
\begin{itemize}
	\item Foram detectados 9 sismos induzidos na região do empreendimento de Machadinho durante o período, sendo 5 próximos à estação BCM2 e 4 próximos à estação MC9. Todos eventos tiveram magnitudes entre -1.0 e 0.0 MLv, eventos pequenos. O maior evento foi detectado em 2023-09-29 10:01:22 (UTC), próximo à estação MC9, com magnitude 0.0 MLv. Não há relatos de eventos que tenham sido sentidos pela população local.  
    \item Foram detectados 2 (dois) desmontes durante o período, sendo o de maior magnitude em 2023-09-06 15:18:24 (UTC) com magnitude 1.7 MLv. Todos desmontes detectados ocorreram longe da região do reservatório (incluindo o de maior magnitude). 
    \item Os parâmetros sísmicos dos eventos detectados são detalhados na Tabela 1. O gráfico de completeza dos dados para as estações BCM2 e MC9 no mês de setembro/2023 é mostrado na Figura 1. Os epicentros dos eventos detectados são mostrados na Figura 2. 
    \item Não foram detectados sismos naturais locais, regionais e/ou telessismos em território brasileiro durante o período englobado por esse relatório. 
    \item O funcionamento das estações BCM2 e MC9 foi adequado no mês de setembro/2023. A estação MC9 foi reinstalada por equipe ténica do IPT e da ENGIE em campo no dia 26/09/2023. O antigo digitalizador da estação foi devolvido à sede do Consórcio Machadinho em Piratuba -- SC. O técnico de campo do consórcio foi treinado para a aquisição dos dados utilizando os novos equipamentos. A estação mostrou funcionamento normal durante a primeira coleta. Sendo assim, a estação MC9 se encontra reinstalada e operacional.
\end{itemize}
    

\section{CONSIDERAÇÕES}
\label{sec:consideracoes}
Continuam válidas as considerações e orientações anteriores a respeito das medidas a serem tomadas em caso ocorrência de um sismo local sentido pela população, i.e., coletar os relatos da população local através de questionários macrossísmicos, contactar a defesa civil para avaliar possíveis danos em estruturas e fornecer orientações e informações à população. 

\assinaturaLucas
\clearpage
\newpage


\clearpage
\newpage

\section{COMPLETUDE DOS DADOS}

\begin{figure}[htb!]
    \centering
	\captionsetup{justification=raggedright, singlelinecheck=false, width=1\textwidth}
    \caption{Gráfico de completude dos dados para o mês de setembro/2023 para as estações BCM2 e MC9.}
    \begin{mdframed}[
        linecolor=black,
        linewidth=1pt,
        roundcorner=10pt,
    ]
    \includegraphics[width=1.0\textwidth]{./machadinho/figuras/completude_BCM2.png} % Substitua pelo nome da imagem e ajuste o tamanho
    \includegraphics[width=1.0\textwidth]{./machadinho/figuras/completude_MC9.png}
    \end{mdframed}
    \caption*{Fonte: IPT}
    \label{fig:completude}
\end{figure}



\section{TABELA DE EVENTOS}
\begin{center}
\scriptsize
\setlength{\arrayrulewidth}{0.05pt}
\begin{longtable}{ccccS[table-format=6.0]S[table-format=7.0]ccc}
\captionsetup{justification=justified,singlelinecheck=false}
\caption{Listagem de eventos detectados e categorizados durante o período de interesse.\\ A coluna \textit{Cat} representaria a categoria na qual o evento foi classificado sendo \textit{Q} = Detonação/Desmontes, \textit{E} = Sismo Regional e \textit{I} = Sismo induzido e \textit{N} = Não-localizável. O valor da energia para os sismos foi obtido a partir da magnitude através da relação proposta por Richter (1958). Fonte: IPT.}\\
%%%%%%%%%%%%%%%%%%%%%%%%%%%%%%%%%%%%%%%%%%%%%%%%%%%%%%%
\hline \\[-4ex]
\hline \\[-5ex]
\multicolumn{1}{c}{ID} &
\multicolumn{1}{c}{Hora de Origem (UTC)} &
\multicolumn{1}{c}{Longitude} &
\multicolumn{1}{c}{Latitude} &
\multicolumn{1}{c}{UTM X} &
\multicolumn{1}{c}{UTM Y} &
\multicolumn{1}{c}{MLv} &
\multicolumn{1}{c}{Energia} &
\multicolumn{1}{c}{Cat} \\


\\[-5.0ex] \hline
\\[-5.0ex]

\multicolumn{1}{c}{\textit{{}}} & 
\multicolumn{1}{c}{\textit{{}}} & 
\multicolumn{1}{c}{\textit{(\textdegree\hspace{0.25em})}} & 
\multicolumn{1}{c}{\textit{(\textdegree\hspace{0.25em})}} & 
\multicolumn{1}{c}{\textit{{(m)}}} & 
\multicolumn{1}{c}{\textit{{(m)}}} & 
\multicolumn{1}{c}{\textit{{}}} & 
\multicolumn{1}{c}{\textit{{(J)}}} & 
\multicolumn{1}{c}{\textit{{}}} \\ 

\\[-5.0ex] \hline
\\[-4.0ex]
\endfirsthead


%%%%%%%%%%%%%%%%%%%%%%%%%%%%%%%%%%%%%%%%%%%%%%%%%%%%%%%
\hline \\[-4ex]
\hline \\[-5ex]
\multicolumn{1}{c}{ID} &
\multicolumn{1}{c}{Hora de Origem (UTC)} &
\multicolumn{1}{c}{Longitude} &
\multicolumn{1}{c}{Latitude} &
\multicolumn{1}{c}{UTM X} &
\multicolumn{1}{c}{UTM Y} &
\multicolumn{1}{c}{MLv} &
\multicolumn{1}{c}{Energia} &
\multicolumn{1}{c}{Cat} \\


\\[-5.0ex] \hline
\\[-5.0ex]

\multicolumn{1}{c}{\textit{{}}} & 
\multicolumn{1}{c}{\textit{{}}} & 
\multicolumn{1}{c}{\textit{(\textdegree\hspace{0.25em})}} & 
\multicolumn{1}{c}{\textit{(\textdegree\hspace{0.25em})}} & 
\multicolumn{1}{c}{\textit{{(m)}}} & 
\multicolumn{1}{c}{\textit{{(m)}}} & 
\multicolumn{1}{c}{\textit{{}}} & 
\multicolumn{1}{c}{\textit{{(J)}}} & 
\multicolumn{1}{c}{\textit{{}}} \\

\\[-5.0ex] \hline
\\[-4.0ex]
\endhead
\hline
\caption*{Fonte: IPT.}

\endlastfoot
%%%%%%%%%%%%%%%%%%%%%%%%%%%%%%%%%%%%%%%%%%%%%%%%%%%%%%%
MC\_20230929\_145625 & 2023-09-29T14:56:25 & -51,4254 & -27,5585 & 458002 & 6951629 & -0,4 & \num[round-precision=3,round-mode=figures,scientific-notation=true]{138.954} & I \\
MC\_20230929\_111457 & 2023-09-29T11:14:57 & -51,4355 & -27,5698 & 457015 & 6950377 & -0,1 & \num[round-precision=3,round-mode=figures,scientific-notation=true]{563.721} & I \\
MC\_20230929\_103129 & 2023-09-29T10:31:29 & -51,4330 & -27,5814 & 457259 & 6949094 & -0,1 & \num[round-precision=3,round-mode=figures,scientific-notation=true]{432.42} & I \\
MC\_20230929\_100122 & 2023-09-29T10:01:22 & -51,4258 & -27,5814 & 457970 & 6949092 & -0,0 & \num[round-precision=3,round-mode=figures,scientific-notation=true]{690.793} & I \\
MC\_20230927\_124624 & 2023-09-27T12:46:24 & -51,3295 & -27,6895 & 467507 & 6937149 & -0,4 & \num[round-precision=3,round-mode=figures,scientific-notation=true]{127.28} & I \\
MC\_20230919\_025258 & 2023-09-19T02:52:58 & -51,3256 & -27,7085 & 467895 & 6935043 & -1,0 & \num[round-precision=3,round-mode=figures,scientific-notation=true]{8.46601} & I \\
MC\_20230916\_152149 & 2023-09-16T15:21:49 & -50,5896 & -27,9284 & 540376 & 6910658 & 1,1 & \num[round-precision=3,round-mode=figures,scientific-notation=true]{82262.6} & Q \\
MC\_20230906\_151824 & 2023-09-06T15:18:24 & -51,8444 & -28,0489 & 417013 & 6897088 & 1,7 & \num[round-precision=3,round-mode=figures,scientific-notation=true]{1.17607e+06} & Q \\
MC\_20230906\_121305 & 2023-09-06T12:13:05 & -51,3376 & -27,7136 & 466720 & 6934480 & -0,6 & \num[round-precision=3,round-mode=figures,scientific-notation=true]{60.8546} & I \\
MC\_20230904\_002008 & 2023-09-04T00:20:08 & -51,3298 & -27,7133 & 467484 & 6934510 & -0,8 & \num[round-precision=3,round-mode=figures,scientific-notation=true]{23.2406} & I \\
MC\_20230902\_081604 & 2023-09-02T08:16:04 & -51,3007 & -27,7044 & 470355 & 6935509 & -0,7 & \num[round-precision=3,round-mode=figures,scientific-notation=true]{30.4004} & I \\
\end{longtable}
\end{center}

\newpage


\begin{figure}[h]
    \centering
	\captionsetup{justification=raggedright, singlelinecheck=false, width=1\textwidth}
    \caption{Mapa de eventos.}
    \includegraphics[width=1.0\textwidth]{/home/ipt/projetos/ipt-latex/boletim/machadinho/figuras/mapaevents.png}
    \caption*{Fonte:IPT}
\end{figure}

\newpage

% Bibliografia
\section{REFERÊNCIAS BIBLIOGRÁFICAS}
\renewcommand{\refname}{REFERÊNCIAS}
\addcontentsline{toc}{section}{REFERÊNCIAS}
\bibliography{ref}


\end{document}
