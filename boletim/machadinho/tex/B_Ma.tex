\documentclass[12pt]{iptex}
% IDIOMA
% Se necessário, inserir línguas e indicar língua principal (main)
\usepackage[main=brazil,english]{babel}
\usepackage[italic]{mathastext}
\usepackage{siunitx}
\sisetup{
    group-separator={.},
    group-minimum-digits=3,
    output-decimal-marker={,}}

% Alterando nome do TOC (adicionar em outras línguas, se necessário)
\addto\captionsbrazil{%
	\renewcommand{\contentsname}{SUMÁRIO}
	\renewcommand{\refname}{REFERÊNCIAS}
}
\addto\captionsenglish{%
	\renewcommand{\contentsname}{CONTENTS}
	\renewcommand{\refname}{REFERENCES}
}
% BIBLIOGRAFIA
\usepackage[abnt-emphasize=bf,alf]{abntex2cite}
% Para bibliografia ABNT com números
%\usepackage[abnt-emphasize=bf,num]{abntex2cite}
%\citebrackets[]

% Para outros estilos o autor deve definir o \bibliographystyle dentro do documento

% Ajustes adicionais
%\setlist[enumerate]{itemsep=3mm} % espaçamento enumerate
%\setlist[itemize]{itemsep=3mm} % espaçamento itemize

% Início do documento
\begin{document}

% CAPA 
\input{B_Ma_capa.tex}

\input{B_Ma_Corpo.tex}

\section{COMPLETUDE DOS DADOS}

\label{fig:completude}
\begin{figure}[htb!]
    \centering
	\captionsetup{justification=raggedright, singlelinecheck=false, width=1\textwidth}
    \caption{Gráfico de completude dos dados para o mês de MÊS para estação ESTAÇÃO.}
    \includegraphics[width=1.0\textwidth]{figuras/completude.png} % Substitua pelo nome da imagem e ajuste o tamanho
    \caption*{Fonte:IPT}
	\label{fig:completude}
\end{figure}



\section{TABELA DE EVENTOS}
\label{sec:tabelas}
\input{B_Ma_Tabela.tex}

\section{MAPA DE EVENTOS}
\label{sec:mapa}
\begin{figure}[ht]
    \centering
	\captionsetup{justification=raggedright, singlelinecheck=false, width=1\textwidth}
    \caption{Mapa de eventos.}
    \includegraphics[width=1.0\textwidth]{figuras/mapaevents.png} % Substitua pelo nome do arquivo de imagem e ajuste o tamanho
    \caption*{Fonte:IPT}
\end{figure}



%\section{REFERÊNCIAS BIBLIOGRÁFICAS}
% Bibliografia
%\renewcommand{\refname}{REFERÊNCIAS}
%\addcontentsline{toc}{section}{REFERÊNCIAS}
%\bibliography{ref}


\end{document}
