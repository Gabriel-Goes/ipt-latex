% Corpo do documento
\section{ÚLTIMOS RELATÓRIOS TÉCNICOS}
\label{sec:ultimos_relatorios}
\begin{itemize}
    \item Relatório Síntese UHMC 2023: Monitoramento sismológico na área do reservatório de Aproveitamento Hidrelétrico de Machadinho, SC/RS, emitido em abril de 2023.
    \item Relatório IPT Nº 205 166 666-1 - “Análise dos registros obtidos entre 01 de dezembro de 2019 e 31 de dezembro de 2021 na rede Sismológica de Itá/Machadinho, RSIM, SC/RS.”, emitido em novembro de 2022.
\end{itemize}

\section{ATIVIDADES REALIZADAS}
\label{sec:atividade}
\begin{itemize}
    \item Encaminhamento do Boletim sísmico nº 25/48-2024, Junho-2023;
    \item Coleta de dados em 01/06/2023 (28/04/2023 a 01/06/2023) e envio dos mesmos para análise no IPT;
    \item Para o período, não houve acesso ao plano de fogo da obra PCH Tupitinga e das pedreiras Engenhos, Kerbermix e PlanaTerra;
    \item Análise preliminar do período que inclui a coleta BCM223118 (31/03/2023 a 28/04/2023) e BCM223152 (28/04/2023 a 01/06/2023); e
    \item Elaboração de gráfico de completeza dos dados, tabela contendo os registros de eventos/detonações detectados.
\end{itemize}

\section{RESULTADOS}
\label{sec:resultados}
Foi detectado um único sismo induzido na região do empreendimento de Machadinho durante o período, na região do remanso do reservatório, com magnitude -0.5 MLv, evento pequeno, em 2023-05-21 21:54:53 (UTC). Não há relatos de eventos que tenham sido sentidos pela população local.

Foram detectados 4 (quatro) desmontes durante o período, sendo o de maior magnitude em 2023-05-19 16:05:43 (UTC) com magnitude 2.0 MLv. Três dos desmontes ocorreram longe da região do reservatório (incluindo o de maior magnitude) e um próximo à cidade de Campos Novos – SC.

Não foram detectados sismos naturais regionais e/ou telessismos no território brasileiro durante o período englobado por este boletim na estação BCM2.

Os parâmetros sísmicos dos eventos detectados são detalhados na Tabela 1. O gráfico de completeza dos dados para a estação BCM2 no mês de maio/2023 é mostrado na Figura 1.

O funcionamento da estação BCM2 foi adequado no mês de maio/2023. A estação MC9 se encontra avariada, conforme detalhado no boletim sísmico Nº 38/48-2021 Jul.20. O digitalizador da estação se encontra na sede do IPT em São Paulo. Recomendações para resumir o funcionamento da estação já foram repassadas pelo IPT à ENGIE, e a empresa já iniciou o processo de aquisição de novos equipamentos.

\section{CONSIDERAÇÕES}
\label{sec:consideracoes}
Continuam válidas as considerações e orientações anteriores a respeito das medidas a serem tomadas em caso ocorrência de um sismo local sentido pela população, i.e., coletar os relatos da população local através de questionários macrossísmicos, contactar a defesa civil para avaliar possíveis danos em estruturas e fornecer orientações e informações à população.

A estação MC9, conforme discutido em boletim anterior, não está operando no momento. Recomendações para resumir o funcionamento da estação já foram repassadas pelo IPT à ENGIE, e a empresa já iniciou o processo de aquisição de novos equipamentos.

%\datahoje{São Paulo, 25 de agosto de 2023}

INSERIR ASSINATURA!
INSERIR ASSINATURA!
INSERIR ASSINATURA!
INSERIR ASSINATURA!
INSERIR ASSINATURA!
INSERIR ASSINATURA!

