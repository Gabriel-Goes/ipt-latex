\documentclass[12pt]{iptex}
% IDIOMA
% Se necessário, inserir línguas e indicar língua principal (main)
\usepackage[main=brazil,english]{babel}

\usepackage{siunitx}
\sisetup{
    group-separator={.},
    group-minimum-digits=3,
    output-decimal-marker={,}}

% Alterando nome do TOC (adicionar em outras línguas, se necessário)
%\addto\captionsbrazil{%
%	\renewcommand{\contentsname}{SUMÁRIO}
%	\renewcommand{\refname}{REFERÊNCIAS}
%}
%\addto\captionsenglish{%
%	\renewcommand{\contentsname}{CONTENTS}
%	\renewcommand{\refname}{REFERENCES}
%}

% BIBLIOGRAFIA
\usepackage[abnt-emphasize=bf,alf]{abntex2cite}

% Para bibliografia ABNT com números
%\usepackage[abnt-emphasize=bf,num]{abntex2cite}
%\citebrackets[]

% Para outros estilos o autor deve definir o \bibliographystyle dentro do documento

% Ajustes adicionais
%\setlist[enumerate]{itemsep=3mm} % espaçamento enumerate
%\setlist[itemize]{itemsep=3mm} % espaçamento itemize

% Início do documento
\begin{document}

% Parâmetros
%\docNum{}
\tipo{BOLETIM SISMOLÓGICiO}
%\cancelaDoc{Relatório Técnico nº 123 456-205} % Descomente para opção cancela e substitui
%\cliente{Instituto de Pesquisas Tecnológicas}{IPT}{Av. Professor Almeida Prado, 532}{55555-555}{São Paulo}{SP}
\data{2023}

\titulo{\textbf{RSIS - Rede Sismológica Itá/Machadinho, SC/RS} \\
\textbf{Reservatório de Machadinho, SC/RS} \\
\textbf{BOLETIM SÍSMICO Nº XXXXXX}}
\unidade{Cidades Infraestruturas e Meio Ambiente}{CIMA}
\lab{Seção de Obras Civis}
\periodo{MES/ANO}{MES/ANO}


% Inserir capa
	\capa
	\pagestyle{timbrado}

% Inserir resumo
	%\resumo[]{Teste}{}

\tableofcontents
\vspace{0.5cm}

\pagebreak
\renewcommand{\thepage}{\arabic{page}}
\setcounter{page}{2}

%\thispagestyle{geral}

\pagestyle{geral}

% Corpo do documento
\section{ÚLTIMOS RELATÓRIOS TÉCNICOS}
\label{sec:ultimos_relatorios}
\begin{itemize}
    \item Relatório Síntese UHMC 2023: Monitoramento sismológico na área do reservatório de Aproveitamento Hidrelétrico de Machadinho, SC/RS, emitido em abril de 2023.
    \item Relatório IPT Nº 205 166 666-1 - “Análise dos registros obtidos entre 01 de dezembro de 2019 e 31 de dezembro de 2021 na rede Sismológica de Itá/Machadinho, RSIM, SC/RS.”, emitido em novembro de 2022.
\end{itemize}

\section{ATIVIDADES REALIZADAS}
\label{sec:atividade}
\begin{itemize}
    \item Encaminhamento do Boletim sísmico nº 25/48-2024, Junho-2023;
    \item Coleta de dados em 01/06/2023 (28/04/2023 a 01/06/2023) e envio dos mesmos para análise no IPT;
    \item Para o período, não houve acesso ao plano de fogo da obra PCH Tupitinga e das pedreiras Engenhos, Kerbermix e PlanaTerra;
    \item Análise preliminar do período que inclui a coleta BCM223118 (31/03/2023 a 28/04/2023) e BCM223152 (28/04/2023 a 01/06/2023); e
    \item Elaboração de gráfico de completeza dos dados, tabela contendo os registros de eventos/detonações detectados.
\end{itemize}

\section{RESULTADOS}
\label{sec:resultados}
Foi detectado um único sismo induzido na região do empreendimento de Machadinho durante o período, na região do remanso do reservatório, com magnitude -0.5 MLv, evento pequeno, em 2023-05-21 21:54:53 (UTC). Não há relatos de eventos que tenham sido sentidos pela população local.

Foram detectados 4 (quatro) desmontes durante o período, sendo o de maior magnitude em 2023-05-19 16:05:43 (UTC) com magnitude 2.0 MLv. Três dos desmontes ocorreram longe da região do reservatório (incluindo o de maior magnitude) e um próximo à cidade de Campos Novos – SC.

Não foram detectados sismos naturais regionais e/ou telessismos no território brasileiro durante o período englobado por este boletim na estação BCM2.

Os parâmetros sísmicos dos eventos detectados são detalhados na Tabela 1. O gráfico de completeza dos dados para a estação BCM2 no mês de maio/2023 é mostrado na Figura 1.

O funcionamento da estação BCM2 foi adequado no mês de maio/2023. A estação MC9 se encontra avariada, conforme detalhado no boletim sísmico Nº 38/48-2021 Jul.20. O digitalizador da estação se encontra na sede do IPT em São Paulo. Recomendações para resumir o funcionamento da estação já foram repassadas pelo IPT à ENGIE, e a empresa já iniciou o processo de aquisição de novos equipamentos.

\section{CONSIDERAÇÕES}
\label{sec:consideracoes}
Continuam válidas as considerações e orientações anteriores a respeito das medidas a serem tomadas em caso ocorrência de um sismo local sentido pela população, i.e., coletar os relatos da população local através de questionários macrossísmicos, contactar a defesa civil para avaliar possíveis danos em estruturas e fornecer orientações e informações à população.

A estação MC9, conforme discutido em boletim anterior, não está operando no momento. Recomendações para resumir o funcionamento da estação já foram repassadas pelo IPT à ENGIE, e a empresa já iniciou o processo de aquisição de novos equipamentos.

%\datahoje{São Paulo, 25 de agosto de 2023}

\newpage
\section{COMPLETUDE DOS DADOS}
\label{fig:completude}

\begin{figure}[htb!]
    \centering
	\captionsetup{justification=raggedright,
                  singlelinecheck=false,
                  width=0.4\textwidth,
                  format=plain}
    \caption{Gráfico de completude dos dados para o mês de MÊS para estação ESTAÇÃO.}
    \includegraphics[width=1.0\textwidth]{../figuras/completude.png} % Substitua pelo nome da imagem e ajuste o tamanho
    \caption*{Fonte:IPT}
	\label{fig:logo}
\end{figure}

\section{TABELA DE EVENTOS}
\label{sec:tabelas}

\input{./Tabela.tex}

\newpage

\section{MAPA DE EVENTOS}
\label{sec:mapa}


\begin{figure}[h]
    \centering
    \caption{Mapa de eventos.}
    \includegraphics[width=1.0\textwidth]{../figuras/mapaevents.png} % Substitua pelo nome do arquivo de imagem e ajuste o tamanho
    \caption*{Fonte:IPT}
\end{figure}



%\section{REFERÊNCIAS BIBLIOGRÁFICAS}
% Bibliografia

%\renewcommand{\refname}{REFERÊNCIAS}
%\addcontentsline{toc}{section}{REFERÊNCIAS}
%\bibliography{ref}

%\label{pag:fim}

%\pagebreak
%\clearpage

% ANEXO A

\end{document}
