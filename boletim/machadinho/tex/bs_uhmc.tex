\documentclass[12]{iptex}
\usepackage{underscore}
\usepackage{geometry}
\usepackage{setspace}
\usepackage{titlesec}
\usepackage{titletoc}
\usepackage{tocloft}
\usepackage{booktabs}
\usepackage{adjustbox}
\usepackage{hyperref}
\usepackage{fancyhdr}
\usepackage{lipsum}
\usepackage{romanbar}
\usepackage{etoolbox}
\usepackage{lastpage}
\usepackage{csvsimple}
\usepackage[bottom]{footmisc}
\usepackage{siunitx}
\usepackage{numprint}
\sisetup{round-mode=places, round-precision=0}

% Configurações de Fonte
\renewcommand{\rmdefault}{phv} % Arial
\renewcommand{\sfdefault}{phv} % Arial
\geometry{a4paper, left=2.5cm, right=2cm, top=4.8cm, bottom=3.5cm}
\renewcommand{\baselinestretch}{1.5}

% Configuração do espaçamento antes e depois de títulos
\titlespacing*{\section}{0pt}{1.5\baselineskip}{0.6\baselineskip}
\titlespacing*{\subsection}{0pt}{1.5\baselineskip}{0.6\baselineskip}
\titlespacing*{\subsubsection}{0pt}{1.5\baselineskip}{0.6\baselineskip}
\titlespacing*{\paragraph}{0pt}{1.5\baselineskip}{0.6\baselineskip}
\titlespacing*{\subparagraph}{0pt}{0pt}{0.6\baselineskip}
\titleformat{\section}[hang]{\bfseries\Large}{\thesection}{1em}{} % Ajusta o alinhamento à esquerda

% Configuração de formatação de listas
%\renewcommand{\labelitemi}{\textbullet}
%\pagestyle{fancy}
%\fancyhf{} % Limpa os cabeçalhos e rodapés
%\fancyhead[L]{\vspace*{1.5cm}\includegraphics[width=4cm]{../figuras/Picture1.png}} % Imagem no cabeçalho
%\fancyfoot[R]{\vspace*{0cm}\includegraphics[width=8cm]{../figuras/Picture2.png}} % Imagem no rodapé
%\renewcommand{\headrulewidth}{0pt} % Remove a linha horizontal do cabeçalho
%\renewcommand{\footrulewidth}{0pt} % Remove a linha horizontal do rodapé

    \pagestyle{timbrado}




\begin{document}
\thispagestyle{plain} % Aplica o estilo de página customizado à primeira página
\title{\textbf{MONITORAMENTO SISMOLÓGICO} \\
\textbf{RSIS - Rede Sismológica Itá/Machadinho, SC/RS} \\
\textbf{Reservatório de Machadinho, SC/RS} \\
\textbf{BOLETIM SÍSMICO Nº XXXXXX}}
\maketitle

\section{PERÍODO DE ANÁLISE:}

\section{ÚLTIMOS RELATÓRIOS TÉCNICOS:}

\section{ATIVIDADES REALIZADAS:}

\section{RESULTADOS:}

\section{CONSIDERAÇÕES:}

\newpage

\section{COMPLETUDE DOS DADOS}

\begin{figure}[h]
    \centering
    \caption{Gráfico de completude dos dados para o mês de XXX para estação XXX.}
    \includegraphics[width=1.0\textwidth]{../figuras/completude.png} % Substitua pelo nome da imagem e ajuste o tamanho
    \caption*{Fonte:IPT}
\end{figure}


\section{TABELA DE EVENTOS}
\input{Tabela.tex}
\newpage


\section{MAPA DE EVENTOS}

\begin{figure}[h]
    \centering
    \caption{Mapa de eventos.}
    \includegraphics[width=1.0\textwidth]{../figuras/mapaevents.png} % Substitua pelo nome do arquivo de imagem e ajuste o tamanho
    \caption*{Fonte:IPT}
\end{figure}

\newpage

\begin{thebibliography}{9}
  \bibitem{ref1} Autor1. \emph{Título da Referência 1}. Editora, Ano.
  \bibitem{ref2} Autor2. \emph{Título da Referência 2}. Editora, Ano.
  % Adicione suas referências bibliográficas aqui
\end{thebibliography}

\end{document}

