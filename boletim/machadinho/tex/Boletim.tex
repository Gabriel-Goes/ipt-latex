\documentclass[12pt]{iptex}
% IDIOMA
% Se necessário, inserir línguas e indicar língua principal (main)
\usepackage[main=brazil,english]{babel}

% Alterando nome do TOC (adicionar em outras línguas, se necessário)
\addto\captionsbrazil{%
	\renewcommand{\contentsname}{SUMÁRIO}
	\renewcommand{\refname}{REFERÊNCIAS}
}
\addto\captionsenglish{%
	\renewcommand{\contentsname}{CONTENTS}
	\renewcommand{\refname}{REFERENCES}
}

% BIBLIOGRAFIA
\usepackage[abnt-emphasize=bf,alf]{abntex2cite}

% Para bibliografia ABNT com números
%\usepackage[abnt-emphasize=bf,num]{abntex2cite}
%\citebrackets[]

% Para outros estilos o autor deve definir o \bibliographystyle dentro do documento

% Ajustes adicionais
%\setlist[enumerate]{itemsep=3mm} % espaçamento enumerate
%\setlist[itemize]{itemsep=3mm} % espaçamento itemize

% Início do documento
\begin{document}

% Parâmetros
%\docNum{}
\tipo{MONITORAMENTO SISMOLÓGICiO}
%\cancelaDoc{Relatório Técnico nº 123 456-205} % Descomente para opção cancela e substitui
%\cliente{Instituto de Pesquisas Tecnológicas}{IPT}{Av. Professor Almeida Prado, 532}{55555-555}{São Paulo}{SP}
\data{2023}

\titulo{\textbf{RSIS - Rede Sismológica Itá/Machadinho, SC/RS} \\
\textbf{Reservatório de Machadinho, SC/RS} \\
\textbf{BOLETIM SÍSMICO Nº XXXXXX}}
\unidade{Cidades Infraestruturas e Meio Ambiente}{CIMA}
\lab{Seção de Obras Civis}
\periodo{MES/ANO}{MES/ANO}

% Inserir capa
	\capa
	\pagestyle{timbrado}

% Inserir resumo
	%\resumo[]{Teste}{}

\tableofcontents
\vspace{0.5cm}

\pagebreak
\renewcommand{\thepage}{\arabic{page}}
\setcounter{page}{2}

%\thispagestyle{geral}

\pagestyle{geral}

% Corpo do documento
\section{ÚLTIMOS RELATÓRIOS TÉCNICOS}
\label{sec:ultimos_relatorios}
\begin{itemize}
    \item Relatório Síntese UHMC 2023: Monitoramento sismológico na área do reservatório de Aproveitamento Hidrelétrico de Machadinho, SC/RS, emitido em abril de 2023.
    \item Relatório IPT Nº 205 166 666-1 - “Análise dos registros obtidos entre 01 de dezembro de 2019 e 31 de dezembro de 2021 na rede Sismológica de Itá/Machadinho, RSIM, SC/RS.”, emitido em novembro de 2022.
\end{itemize}

\section{ATIVIDADES REALIZADAS}
\label{sec:atividade}
\begin{itemize}
    \item Encaminhamento do Boletim sísmico nº 25/48-2024, Junho-2023;
    \item Coleta de dados em 01/06/2023 (28/04/2023 a 01/06/2023) e envio dos mesmos para análise no IPT;
    \item Para o período, não houve acesso ao plano de fogo da obra PCH Tupitinga e das pedreiras Engenhos, Kerbermix e PlanaTerra;
    \item Análise preliminar do período que inclui a coleta BCM223118 (31/03/2023 a 28/04/2023) e BCM223152 (28/04/2023 a 01/06/2023); e
    \item Elaboração de gráfico de completeza dos dados, tabela contendo os registros de eventos/detonações detectados.
\end{itemize}

\section{RESULTADOS}
\label{sec:resultados}
Foi detectado um único sismo induzido na região do empreendimento de Machadinho durante o período, na região do remanso do reservatório, com magnitude -0.5 MLv, evento pequeno, em 2023-05-21 21:54:53 (UTC). Não há relatos de eventos que tenham sido sentidos pela população local.

Foram detectados 4 (quatro) desmontes durante o período, sendo o de maior magnitude em 2023-05-19 16:05:43 (UTC) com magnitude 2.0 MLv. Três dos desmontes ocorreram longe da região do reservatório (incluindo o de maior magnitude) e um próximo à cidade de Campos Novos – SC.

Não foram detectados sismos naturais regionais e/ou telessismos no território brasileiro durante o período englobado por este boletim na estação BCM2.

Os parâmetros sísmicos dos eventos detectados são detalhados na Tabela 1. O gráfico de completeza dos dados para a estação BCM2 no mês de maio/2023 é mostrado na Figura 1.

O funcionamento da estação BCM2 foi adequado no mês de maio/2023. A estação MC9 se encontra avariada, conforme detalhado no boletim sísmico Nº 38/48-2021 Jul.20. O digitalizador da estação se encontra na sede do IPT em São Paulo. Recomendações para resumir o funcionamento da estação já foram repassadas pelo IPT à ENGIE, e a empresa já iniciou o processo de aquisição de novos equipamentos.

\section{CONSIDERAÇÕES}
\label{sec:consideracoes}
Continuam válidas as considerações e orientações anteriores a respeito das medidas a serem tomadas em caso ocorrência de um sismo local sentido pela população, i.e., coletar os relatos da população local através de questionários macrossísmicos, contactar a defesa civil para avaliar possíveis danos em estruturas e fornecer orientações e informações à população.

A estação MC9, conforme discutido em boletim anterior, não está operando no momento. Recomendações para resumir o funcionamento da estação já foram repassadas pelo IPT à ENGIE, e a empresa já iniciou o processo de aquisição de novos equipamentos.

%\datahoje{São Paulo, 25 de agosto de 2023}

\section{ESTRUTURA}
\label{sec:estrutura}

Vamos considerar o código simples abaixo para entender um pouco melhor a estrutura de um documento escrito em \LaTeX.

\small\begin{verbatim}
	\documentclass[a4paper,12pt]{article} % Classe do documento com opções
	\usepackage[T1]{fontenc} % Pacotes utilizados

	% Início do documento
	\begin{document}

		Aqui escrevemos o conteúdo do documento

	\end{document}
\end{verbatim}

Nota-se um ambiente delimitado por \verb|\begin{document}....\end{document}|. É nesse local que todos os elementos textuais devem ser alocados (capa, resumo, sumário, seções, bibliografias e anexos). Os elementos anteriores a \verb|\begin{document}| compõem o preâmbulo do documento. É neste lugar em que os arquivos de classe e pacotes são chamados, em que comandos são criados e em que parâmetros de forma são modificados.


\section{SEÇÕES E SUBSEÇÕES}
\label{sec:secoes}

As seções podem ser criadas com os comandos mostrados abaixo. Note que o título do primeiro nível deve ser escrito em caixa alta para que apareça corretamente no Sumário. Além disso, no quarto nível deve-se incluir \verb|\hspace{0pt} \\|, caso contrário não há quebra de linha entre o título da subseção e o texto.

\begin{itemize}
	\item \verb|\section{TÍTULO}| (primeiro nível);
	\item \verb|\subsection{Título}| (segundo nível);
	\item \verb|\subsubsection{Título}| (terceiro nível); e
	\item \verb|\paragraph{Título} \hspace{0pt} \\| (quarto nível).
\end{itemize}


\section{PRIMEIRO NÍVEL}
\label{sec:primeiro_nivel}

\subsection{Segundo nível}
\label{subsec:segundo_nivel}

\subsubsection{Terceiro nível}
\label{subsubsec:terceiro_nivel}

\paragraph{Quarto nível} \hspace{0pt} \\

\section{REFERÊNCIAS INTERNAS}
\label{sec:ref_interna}

As referências internas são feitas com o par \verb|\label{}| (logo após o elemento a ser referenciado), \verb|\ref{}| (ao citar o elemento no texto). Qualquer nome pode ser utilizado entre chaves como referência. Para referenciar o item~\ref{subsubsec:terceiro_nivel} devemos escrever:

\begin{verbatim}
	item~\ref{subsubsec:terceiro_nivel}
\end{verbatim}

\noindent sendo que o $\sim$ indica que queremos forçar a palavra ``item'' e a referência \ref{subsubsec:terceiro_nivel} na mesma linha.

\section{LISTAS}
\label{sec:listas}

Em \LaTeX~podemos criar listas não ordenadas, com o ambiente \textit{itemize} ou listas ordenadas, com o ambiente \textit{enumerate}. Para mais informações, vide a \href{https://www.overleaf.com/learn/latex/Lists}{\textcolor{blue}{documentação da página do Overleaf}}.

\subsection{Listas não ordenadas}
\label{subsec:itemize}

Uma lista não ordenada é criada com o ambiente\textit{itemize}. Pode-se controlar a separação entre os itens com o parâmetro opcional \textit{itemsep=2mm} (ex. separação de 2 milímetros). Ademais, pode-se controlar todos os marcadores com o parâmetro opcional \textit{label}, ou então individualmente em cada item.

\begin{itemize}[itemsep=1mm, label=$\bullet$]
	\item Primeiro item.
	\item Segundo item.
	\begin{itemize}[itemsep=1mm]
		\item[$\ast$] Subitem dentro do segundo item.
	\end{itemize}
	\item Terceiro item.
	\begin{itemize}[itemsep=1mm]
		\item[--] Subitem dentro do terceiro item.
	\end{itemize}
\end{itemize}

\subsection{\textit{enumerate}}
\label{subsec:enumerate}

Listas ordenadas podem ser criadas no ambiente\textit{enumerate}. O formato padrão é a utilização de algoritmos arábicos. As possibilidades são dadas a seguir.

\begin{itemize}
	\item \verb|\arabic|
	\item \verb|\roman|
	\item \verb|\Roman|
	\item \verb|\Alph|
	\item \verb|\alph|
\end{itemize}

Com o parâmetro opcional \textit{label} pode-se fazer o controle de como os itens são mostrados. O parâmetro \verb|label=\Roman*)|, por exemplo, nos dá um algarismo romano maiúsculo, seguido de parenteses, ou seja,

\begin{enumerate}[itemsep=1mm, label=\Roman*)]
	\item Primeiro item.
	\item Segundo item.
	\item Terceiro item.
\end{enumerate}


Subitens também podem ser criados da mesma forma que no caso do ambiente \textit{itemize}. Um exemplo é mostrado a seguir.

\begin{enumerate}[itemsep=1mm, label=\Alph*.]
	\item Primeiro item.
	\begin{enumerate}[itemsep=1mm, label=(\roman*)]
		\item Primeiro subitem do primeiro item.
		\item Segundo subitem do primeiro item.
	\end{enumerate}
	\item Segundo item.
	\begin{enumerate}[itemsep=1mm, label=(\roman*)]
		\item Primeiro subitem do segundo item.
		\item Segundo subitem do segundo item.
	\end{enumerate}
\end{enumerate}

\section{FLOATS}
\label{sec:floats}

\textit{Floats} são elementos que não podem ser quebrados entre páginas. Não fazem parte do fluxo normal do texto e ``flutuam'', podendo ser deslocados para a página posterior, por exemplo. Como padrão temos duas classes: figuras (\textit{figure}) e tabelas (\textit{table}). Contudo, novas classes podem ser criadas com o pacote \textit{float}.

Pela característica de ``flutuarem'' no documento, parâmetros posicionais costumam ser utilizados para controlar o posicionamento do elemento de acordo com a necessidade do usuário. Os parâmetros são:

\begin{itemize}
	\item \textbf{t}: na parte superior de uma página de texto;
	\item \textbf{b}: na parte inferior de uma página de texto;
	\item \textbf{h}: na posição do texto em que o \textit{float} aparece (aproximadamente na mesma posição). Caso deseje forçar o elemento exatamente no local inserido, usar \verb|\usepackage{float}| e o parâmetro \textbf{H};
	\item \textbf{p}: em uma página flutuante separada, que não contém texto; e
	\item \textbf{!}: ignora os parâmetros internos que o \LaTeX~usa para determinar o posicionamento do \textit{float}.
\end{itemize}

\subsection{Figuras}
\label{fig:figuras}

A inserção de figuras em \LaTeX~acontece dentro do ambiente \textit{figure}. Figuras são \textit{floats}, ou seja, não podem ser quebradas entre páginas. A Figura~\ref{fig:logo} é um exemplo (se possível, explore o código no arquivo \textit{tex}).

\begin{figure}[htb]
	\centering
	\captionsetup{justification=raggedright, singlelinecheck=false, width=0.4\textwidth}
	\caption{Logomarca do IPT.}
	\includegraphics[width=0.4\textwidth]{./figuras/headImage.pdf}
	\caption*{Fonte: IPT}
	\label{fig:logo}
\end{figure}


\subsection{Tabelas}
\label{subsec:tabelas}

Um exemplo de tabela usando \LaTeX~é dado abaixo, na Tabela~\ref{tab:exemplo}. Explore os elementos da tabela e veja as alterações provocadas no documento compilado.

\begin{table}[htb!]
	\centering
	\caption{Exemplo de tabela usando \LaTeX.}
	\begin{tabular}{c|c|c}
		\hline
		\textbf{Coluna 1} & \textbf{Coluna 2} & \textbf{Coluna 3} \\
		\hline
		a & b & c \\
		d & e & f \\
		g & h & i \\
		\hline
	\end{tabular}
	\caption*{Fonte: IPT}
	\label{tab:exemplo}
\end{table}

\section{Fórmulas}

\LaTeX~é bastante utilizado na produção de textos e artigos científicos em exatas devido à facilidade de inserção de fórmulas e sua qualidade estética. Abaixo temos dois exemplos da equação de \textit{Navier-Stokes}. A equação~\ref{eq:NS_indicial} mostra a forma indicial (notação de Einstein) para o caso de fluido incompressível com viscosidade constante. Já o sistema de equações~\ref{eq:NS_2D} mostra um caso incompressível, bidimensional, com viscosidade e densidade constantes.

\begin{equation}
	\rho\frac{Du_i}{Dt} = \rho f_i - \frac{\partial p}{\partial x_i} + \frac{\partial}{\partial x_j}\left[2\mu\left(e_{ij} - \frac{\Delta\delta_{ij}}{3}\right)\right]
	\label{eq:NS_indicial}
\end{equation}

\begin{equation}
\begin{aligned}
	\rho\left(\frac{\partial u}{\partial t} + u\frac{\partial u}{\partial x} + v\frac{\partial u}{\partial y}\right) &= \mu\left[\frac{\partial^2 u}{\partial x^2} + \frac{\partial^2 u}{\partial y^2}\right] - \frac{\partial p}{\partial x} + \rho g_x \\
	\rho\left(\frac{\partial v}{\partial t} + u\frac{\partial v}{\partial x} + v\frac{\partial v}{\partial y}\right) &= \mu\left[\frac{\partial^2 v}{\partial x^2} + \frac{\partial^2 v}{\partial y^2}\right] - \frac{\partial p}{\partial y} + \rho g_y
\end{aligned}
\label{eq:NS_2D}
\end{equation}

%De acordo com a Equação~\ref{eq:newton2}, $\mathbf{p}$ é a quantidade de movimento. O termo $\upalpha$ e o termo x .... F$_{\text{CDH,ref}}$
%
%\begin{equation}
%	\mathbf{F} = \frac{d\mathbf{p}}{dt} = \frac{d}{dt}(mv)
%	\label{eq:newton2}
%\end{equation}
%
%\begin{equation}
%	y = \alpha x + \beta
%\end{equation}

\clearpage

\subsection{Fórmulas químicas}
\label{subsec:formulas_quimicas}

Existem alguns pacotes disponíveis. Um deles é o \verb|chemfig|. Alguns exemplos de aplicação são dados abaixo. Para mais exemplos, vide artigo no \href{https://www.overleaf.com/learn/latex/Chemistry_formulae}{\textcolor{blue}{Overleaf}}.

\vspace{0.5cm}

% ângulos absolutos
\chemfig{A-[:50]B-[:-25]C}

\vspace{0.5cm}

% Polígonos regulares
\chemfig{A*5(-B-C-D-E=)}

\vspace{0.5cm}

\chemfig{H-C(-[2]H)(-[6]H)-C(=[1]O)(-[7]H)}

\section{REFERÊNCIAS BIBLIOGRÁFICAS}

Para utilizar referências bibliográficas automáticas deve-se crias um arquivo \textit{BibTeX}, tal qual o \textit{ref.bib} que foi fornecido. Para utilizar ao longo do texto utiliza-se os comandos \verb|\cite{}| ou \verb|\citeonline{}|.

Para facilitar a criação do arquivo \textit{BibTeX} pode-se recorrer ao assistente do \textit{TexStudio} (aba Bibliografia), ao \href{https://scholar.google.com/}{\textcolor{blue}{Google Scholar}} ou a uma ferramenta para gestão de referências bibliográficas, tais quais o \href{https://www.jabref.org/}{\textcolor{blue}{JabRef}}, o \href{https://www.mendeley.com/?interaction_required=true}{\textcolor{blue}{Mendeley}}, o \href{https://www.zotero.org/}{\textcolor{blue}{Zotero}} e o \href{https://www.mybib.com/}{\textcolor{blue}{MyBib}}.

\section{APLICAÇÃO DA CLASSE IPTEX}

A classe IPTeX foi criada para facilitar a utilização do \LaTeX em Relatórios Técnicos. Certificados de Calibração e demais documentos do IPT. O desenvolvimento ocorre de forma contínua no repositório \href{https://github.com/iptsp/IPTeX}{\textcolor{blue}{https://github.com/iptsp/IPTeX}} desde janeiro de 2022.

Para detalhes de utilização veja o manual de instruções disponível no repositório e, caso deseje criar um novo documento técnico, siga um dos modelos criados.

% Bibliografia

%\renewcommand{\refname}{REFERÊNCIAS}
%\addcontentsline{toc}{section}{REFERÊNCIAS}
%\bibliography{ref}

%\label{pag:fim}

\pagebreak
\clearpage

% ANEXO A

\end{document}
