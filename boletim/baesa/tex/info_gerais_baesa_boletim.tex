\section{INFORMAÇÕES GERAIS}
A seguir são apresentadas informações gerais dos empreendimentos, do monitoramento sismológico, das condicionantes ambientais e do contrato de execução das atividades desta prestação de serviçoitado ou excluído. especializado na área de Sismologia.

\subsection{Características dos empreendimentos}
Os empreendimentos constituem-se dos Aproveitamentos Hidrelétricos de:
\begin{itemize}
    \item Barra Grande, situado no rio Pelotas, SC/RS, com o barramento na divisa dos municípios de Anita Garibaldi, SC (margem direita) e Pinhal da Serra, RS (margem esquerda). O reservatório ocupa parcialmente terras dos municípios de: Anita Garibaldi, Cerro Negro, Campo Belo do Sul, Capão Alto e Lages, no Estado de Santa Catarina e Pinhal da Serra, Esmeralda, Vacaria e Bom Jesus, no Estado do Rio Grande do Sul; e
    \item Campos Novos, situado no rio Canoas, SC, com o barramento na divisa dos municípios de Campos Novos, SC (margem direita) e Celso Ramos, SC (margem esquerda). O reservatório atinge áreas dos municípios de Abdon Batista, Anita Garibaldi, Campos Novos e Celso Ramos, no Estado de Santa Catarina.
\end{itemize}

A operação, manutenção e administração da UHE Barra Grande e da UHE Campos Novos são de responsabilidade da Energética Barra Grande S. A. e Campos Novos Energia S. A., respectivamente.

As características dos reservatórios, resumidamente, são:
\begin{table}
% CARACTERISTICAS DOS RESERVATÓRIOS
\end{table}
\caption{(*) esvaziamento e enchimento do reservatório em decorrência de problema ocorrido no túnel II de desvio.}


\subsection{Informações sobre as condicionantes ambientais:}
As condicionantes da Licença de Operação referentes ao monitoramento sismológico são:
\begin{itemize}
    \item BAESA: Condicionante 2.1, item e, da Licença Ambiental de Operação no 447/2005, 2a Renovação, emitida em 26/03/2014 pelo Instituto Brasileiro do Meio Ambiente e dos Recursos Naturais Renováveis – IBAMA para a Usina Hidrelétrica Barra Grande, que determina a continuidade do Programa de Monitoramento Sismológico; e
    \item ENERCAN: Licença Ambiental de Operação no 9665/2014, emitida em 23.12.2014 pela Fundação do Meio Ambiente – FATMA do Estado de Santa Catarina para a Usina Hidrelétrica Campos Novos, que determina a execução do Monitoramento das Condições Sismológicas.
\end{itemize}

\subsection{O monitoramento sismológico}
O monitoramento sismológico visa detectar as atividades sísmicas, natural ou induzida, nas áreas de influência dos reservatórios dos Aproveitamentos Hidrelétricos de Barra Grande, SC/RS e de Campos Novos, SC, fornecendo diagnósticos sobre as características da sismicidade local e suas possíveis consequências, possibilitando tomar medidas mitigadoras, atendendo as necessidades previstas nos Programas Ambientais destes empreendimentos.
Os sismos estão agrupados nos Fatores do Meio Físico. Como Indicadores Ambientais serão avaliados: distribuição geográfica (localização dos epicentros), tamanho (magnitude e intensidade) e frequência de ocorrência (distribuição temporal). A análise conjunta dos resultados destes indicadores possibilitará qualificar (natural ou induzida, local ou regional) e quantificar (fraca/média/forte, intermitente/contínua etc.) a sismicidade fornecendo subsídios para outros programas, tais como: Gerenciamento de Riscos e Comunicação Social. Os resultados indicarão a necessidade ou não de uma redefinição do monitoramento sismológico deste estudo, com o intuito de se estudar adequadamente a atividade sísmica local.
O monitoramento local teve início em meados de fevereiro de 2004 com a instalação da Estação “vigilante” BCM2 para auscultar a sismicidade local na fase  prévia ao enchimento dos reservatórios e entre maio-dezembro de 2005 foram instaladas as outras 4 estações, compondo a RSBC – Rede Sismológica de Barra Grande e Campos Novos, para a auscultação nos períodos de enchimento e pós-enchimento dos reservatórios.

A seguir são apresentados os dados referentes à localização das estações da RSBC e as respectivas datas de instalação:
\begin{table}
% TABELA DA LOCALIZAÇÃO
\end{table}
\caption*{(**) em 20.01.2009 foi desativada e os equipamentos retornaram para a estação BC7}
\caption*{(*) em 26.01.2015 foi desativada a estação BC7.}
\caption*{A partir de janeiro de 2015 não estão sendo utilizados mais os dados da Estação BCM2.}

As estações sismológicas e os empreendimentos estão localizados em rochas basálticas toleíticas e riodacitos da bacia do Paraná.
O trabalho atual é uma continuação do monitoramento sismológico em execução na área, através da RSBC composta inicialmente de 5 estações digitais triaxiais de período curto, compreendendo a etapa de pAuditório István Jancsóós-enchimento dos citados reservatórios. Em função das características da sismicidade local, a partir de janeiro de 2015, a RSBC passou a funcionar com 3 estações (BC4, BC9 e BC12).
Cada estação sismológica é composta por registrador digital de 24 bits, sismômetro triaxial de período curto (fo = 1 Hz), ajuste do relógio/localização através de GPS (Global Position System), memórias flash para gravação dos dados e sistema de alimentação através de baterias estacionárias seladas e painéis solares.
No Anexo A, Figura 1, é apresentado o mapa da região de interesse do empreendimento com a localização das estações e eventos no entorno (caso existam) para o período abrangido pelo presente boletim sísmico.
Este estudo também contribuirá com informações sobre a ocorrência de sismos nos Estados de Santa Catarina, do Rio Grande do Sul e regiões vizinhas, contribuindo com dados, melhorando o conhecimento da sismicidade brasileira.
Em função das características operacionais das estações sismológicas e/ou dos eventos sísmicos que venham a ocorrer, serão obtidas também informações sobre atividade sísmica regional e mundial.
No atual estudo, a continuidade do monitoramento sismológico consiste das atividades que basicamente englobam: coleta e envio dos dados, a sua interpretação e a emissão de boletins sísmicos mensais e de relatórios técnicos semestrais, contendo os resultados da análise, considerações sobre a sismicidade e recomendações.

\subsection{O contrato de execução do serviço:}

A Instituição responsável pelo monitoramento sismológico:
Instituto de Pesquisas Tecnológicas do Estado de São Paulo S. A. - IPT
Av. Prof. Almeida Prado, 532 – CEP 05508-901
Cidade Universitária – Butantã – São Paulo – SP
CNPJ: 60.633.674/0001-55
IE: 105.933.432.110
Com relação ao CTF/CR - Cadastro Técnico Federal/Certificado de Regularidade tem-se que:
Responsável
Registro
Validade
Chave de Autenticação
Pessoa
IPT
676518
19/10/2023
Z7FX6S4TNR1QGXE8
Jurídica


E as ARTs – Anotação de Responsabilidade Técnica (CREA- SP):
Empresa
Nº
Emissão
Validade
BAESA
28027230230714974
30.05.2023
30.11.2024
ENERCAN
28027230230711325
30.05.2023
30.11.2024
OBS.: cópias dos CTFs e das ARTs encontram-se apresentadas no final deste Relatório Mensal de Atividade, no Anexo B.


