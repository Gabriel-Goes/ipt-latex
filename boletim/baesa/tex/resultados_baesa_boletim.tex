
\section{RESULTADOS}
Durante o mês de maio.2023 não foram detectados sismos induzidos na vizinhança dos reservatórios de Barra Grande e Campos Novos. Não há relatos de nenhum sismo que tenha sido sentido pela população local.
Foi detectado 1 desmonte em obras/pedreiras no período, com magnitude 1.7 MLv em 2023-05-19 16:06:00 (UTC), distante da região dos reservatórios.
Não foram detectados sismos naturais locais/regionais ou telessismos (sismos com epicentros distantes) no território brasileiro durante o período.
Na Tabela 1 encontram-se descritas as características e os parâmetros epicentrais do evento detectado pela RSBC no mês de maio.2023. As estações BC4, BC9 e BC12 operaram normalmente no período.

\begin{table}[htb!]
\caption{Eventos detectados durante o mês de maio de 2023 na análise dos dados da RSBC.}
\begin{tabular}[ccc]
    1 & 2 & 3 \\
    4 & 5 & 6 \\
\end{tabular}
\caption*{A coluna C indica a categoria do evento onde Q = Detonação, I = Sismo induzido e  E = Sismo natural.}
\end{table}

O funcionamento das estações foi satisfatório para o período, embora se haja constatado problema com a componente de registro Norte-Sul da estação BC9. A completeza dos dados para o período é mostrada na Figura 2, Anexo A.
