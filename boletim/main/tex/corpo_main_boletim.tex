% Corpo do documento
\section{ÚLTIMOS RELATÓRIOS TÉCNICOS}
\label{sec:ultimos_relatorios}
\begin{itemize}
    \item Relatório Síntese UHMC 2023: Monitoramento sismológico na área do reservatório de Aproveitamento Hidrelétrico de Machadinho, SC/RS, emitido em abril de 2023.
    \item Relatório IPT Nº 205 166 666-1 - “Análise dos registros obtidos entre 01 de dezembro de 2019 e 31 de dezembro de 2021 na rede Sismológica de Itá/Machadinho, RSIM, SC/RS.”, emitido em novembro de 2022.
\end{itemize}

\section{ATIVIDADES REALIZADAS}
\label{sec:atividade}
\begin{itemize}
    \item Encaminhamento do Boletim sísmico nº 28/48-2024, Agosto-2023;
    \item Coleta de dados em 01/09/2023 (03/08/2023 a 01/09/2023) e envio dos mesmos para análise no IPT;
    \item Para o período, não houve acesso ao plano de fogo da obra PCH Tupitinga e das pedreiras Engenhos, Kerbermix e PlanaTerra;
    \item Análise preliminar do período que inclui as coletas BCM223215 (03/07/2023 a 03/08/2023) e BCM223244 (03/08/2023 a 01/09/2023); e
    \item Elaboração de gráfico de completeza dos dados, tabela contendo os registros de eventos/detonações detectados.
\end{itemize}

\section{RESULTADOS}
\label{sec:resultados}
Foi detectado um único sismo induzido na região do empreendimento de Machadinho durante o período, na região do remanso do reservatório, próximo à estação BCM2 com magnitude -0.9 MLv, evento pequeno, em 2023-08-09 10:53:18 (UTC). Não há relatos de eventos que tenham sido sentidos pela população local.

Foram detectados 14 (quatorze) desmontes durante o período, sendo o de maior magnitude em 2023-08-01 18:17:15 (UTC) com magnitude 1.9 MLv. Todos os desmontes ocorreram longe da região do reservatório (incluindo o de maior magnitude), com exceção de um desmonte ocorrido ao sul da cidade de Campos Novos – SC.

Foi detectado um sismo natural regional suspeito em 2023-08-15 19:15:41 (UTC) próximo ao distrito de Santa Terezinha do Salto - Lages/SC, com magnitude calculada 1.5 MLv. Não foram detectados telessismos no território brasileiro durante o período englobado por este boletim na estação BCM2.

Os parâmetros sísmicos dos eventos detectados são detalhados na Tabela 1. O gráfico de completeza dos dados para a estação BCM2 no mês de agosto/2023 é mostrado na Figura 1.

O funcionamento da estação BCM2 foi adequado no mês de agosto/2023. A estação MC9, que se encontrava avariada, conforme detalhado no boletim sísmico Nº 38/48-2021 Jul.20, teve seus equipamentos substituídos e foi reinstalada no dia 27/09/2023. O antigo digitalizador da estação foi retornado à sede do Consórcio Machadinho em Piratuba/SC. A estação voltou a funcionar corretamente e seus dados serão incluídos na próxima análise.

\section{CONSIDERAÇÕES}
\label{sec:consideracoes}
Continuam válidas as considerações e orientações anteriores a respeito das medidas a serem tomadas em caso ocorrência de um sismo local sentido pela população, i.e., coletar os relatos da população local através de questionários macrossísmicos, contactar a defesa civil para avaliar possíveis danos em estruturas e fornecer orientações e informações à população.

A estação MC9 voltou a operar normalmente com novos equipamentos em 27/09/2023, e seus dados serão incluídos na próxima análise.

\assinaturaLucas
